%%% ====================================================================
%%% 	theory.tex - OpenPave Theory
%%%   	(c) Openpave.org, 2006.
%%%
%%% $OpenPave$
%%% ====================================================================

\documentclass[11pt,oneside,letterpaper]{optech}
\settrimmedsize{11in}{210mm}{*}
\setlength{\trimtop}{0.38197\stockheight-0.38197\paperheight}
\setlength{\trimedge}{0.5\stockwidth-0.5\paperwidth}
\settypeblocksize{9in-2em}{105mm}{*}
\setulmargins{*}{*}{1.618}
\setheadfoot{1.5em}{2.5em}
\setheaderspaces{*}{1.5em}{*}
\setlrmargins{20mm}{*}{*}
\setmarginnotes{1em}{65mm-1em}{1em}
\checkandfixthelayout
\setlength{\headwidth}{\textwidth}
  \addtolength{\headwidth}{\marginparsep}
  \addtolength{\headwidth}{\marginparwidth}

\aliaspagestyle{chapter}{opchap}
\chapterstyle{optech}
\pagestyle{optech}

\usepackage{latexsym}
\usepackage{amssymb}
\usepackage{amsfonts}
\usepackage[fleqn]{amsmath}
% For various symbols
\usepackage{textcomp}

\ifpdf
\pdfinfo {
	/Title (OpenPave Theory)
	/Subject (The theory behind the OpenPave software)
	/Author (Jeremy D. Lea)
	/Keywords (Pavement design, Layered elastic theory)
}
\fi
\usepackage{lscape,rotating,flafter}

\usepackage{emp}
\empTeX{\documentclass[11pt]{optech}}
\empaddtoTeX{\fontfamily{\sfdefault}
  \fontsize{8}{10}
  \selectfont}
\empprelude{input rboxes}

% Bibliographic reference style.
\usepackage[round,comma,authoryear,sort&compress]{natbib}
\usepackage[nolist]{acronym}

%\DoubleSpacing

\newcommand*{\OP}{\textsc{OpenPave.org}}
\newcommand*{\Fortran}{\textsc{Fortran}}
\newcommand*{\CC}{C\nolinebreak\hspace{-.06em}\raisebox{.5ex}{\tiny\textbf
  +}\nolinebreak\hspace{-.09em}\raisebox{.5ex}{\tiny\textbf +}}

\begin{document}
\begin{empfile}

\begin{titlingpage}
  \null\vfil
  \begin{adjustwidth}{0pt}{\textwidth-\headwidth}
  \begin{center}%
    {\Large \textbf{The theory behind \OP\ software} \par}%
    \vskip 3em%
    {\large
     \lineskip .75em%
        \textbf{Jeremy D. Lea}
      \vskip 3em%
 		}
    \vskip 3em%
    {\lineskip .75em%
        \copyright\ 2006 \OP.
    }
    \vskip 1.5em%
    {\large 10 November 2006 \par}
  \end{center}
  \end{adjustwidth}
  \vfil\null
\end{titlingpage}

\frontmatter
\setsecnumdepth{section}
\settocdepth{section}
\tableofcontents
%\listoffigures
%\listoftables

\mainmatter

\chapter{Introduction}

\OP\ is committed to supplying the best possible open source software for
pavement engineering.  To that end it is important that the theoretical and
mathematical foundation underlying that software are well understood, and
that the mathematics are accurately and faithfully translated into code. 
Thus, this document outlines the mathematics of the software, and how this
is carried down to level of source code.  The source code makes reference to
this document rather than extensive comments, since it is hard to express
complex mathematics in comments.  However, this document is not just for
people looking at the source code -- It should also serve as an excellent
reference for anyone looking to understand the more theoretical aspects of
pavement engineering, and of some associated fields.

This document is written in \AmS -\LaTeX, and is licensed under the same
conditions as the source code, the \ac{ADDL}.  You are, therefore, free to
make alterations to this document, and to use it as course notes or for
other purposes, provided that you distribute any changes to you make. 
Please review the conditions in the license for more information.

\section{Typographical conventions}

Because this document covers a number of diverse fields, there is some need
to standardize the typographical conventions.  As a result there are a
number of places where this document does not follow the `normal'
conventions for a particular field.  However, since there is already
considerable variety in the various fields, it is hoped that readers will be
able to follow without having to completely re-wire their brains.

The following conventions are used in this document:

% XXX: need to invent a convention!

\chapter{Mathematics and Numerical Theory}


\chapter{Statistics and Reliability}

While reliability is a fairly simple concept, there are many different
approaches to calculating it, not all of which provide a consistent,
unbiased analysis.  In this section, a short mathematical background to the
approach used in \OP\ is given.  The approach is an adaptive importance
sampling algorithm, based on assumed distributions for the design
parameters.

\section{Basic introduction to statistical terminology}

Often the terminology used in statistical analysis can be confusing, and
unless the reader understands the implications of the words being used, they
can often loose track of some of the subtleties of the problem.  If the
reader is well versed in statistics, then they probably want to skip this
section.  Readers who have little or no background in statistics will
certainly want to consult an entry level statistics text before continuing.

\subsection{Definition of terminology}

The three basic building blocks of statistics are variables, events and
probabilities.  A variable is a number, which in general can take on a value
within some range, but has a fixed value when actually evaluated.  Variables
can either be discrete (integer) or continuous (real) numbers.  There are
three basic types of variables: constants, parameters and random variables. 
A constant is, fairly obviously, a constant number whose value is known
precisely (normally fairly rare in engineering).  A parameter is a variable
whose value has been determined based on past data, and a random variable
is, well, random -- when measured it could take on any value within it's
range.  However, a random variable can be observed (under a given state of
the system) and the observed value of a random variable is fixed and is
known as an observation.  Observations are often in pairs or sets of random
variables observed under the same state of the system.

Random variables can be classified as exogenous and endogenous, which refer
to variables which are explain the randomness in the system (hence they are
also called explanatory variables) and variables which are explained by the
randomness in the system.  Thus, when a random variable is modeled as a
function of other random variables (i.e. $y=f(\mathbf{x})$), the
$\mathbf{x}$ variables are exogenous, and the $y$ variable endogenous.  The
value of $y$ might be observable, but the randomness this value is assumed
to be explained entirely by the randomness in the explanatory variables.

An event occurs when the observed value of one or more random variables meet
some criteria.  Thus an event is always an endogenous Boolean random
variable, which can be represented as an indicator variable.  The act of
observing an event, to determine an outcome, is a trial.

There are different kinds of randomness in any analysis system.  The first
type is inherent randomness -- things that cannot be control
\textit{mathematically}, such as layer thickness, material properties or
wheel loads.  While these can be controlled physically, they are represented
by random variables whose variability cannot be reduced mathematically.  The
second kind are errors: These are problems while observing the random
variable which result in the observation being different to the real value. 
These errors can be random -- they have some variance but zero mean, or
systematic -- they have a non-zero mean.  The randomness can also be due to
a lack of data -- the more observations which have been made, the more is
known about the system.  The first two kinds of randomness tend to affect
the individual outcomes, while the third tends to affect the parameters.

Any parameters in the model are endogenous, because, although they might be
constants, they are unknown, and we have to approximate their values using a
random variable.  Thus parameters are treated as constants in some
situations and random variables in others.

Random variables in the same set, measured under the same state, can be
dependent or independent.  If they are dependent, then the value of one of
the variables tells you something about the value of another.  This is also
known as correlation, although this terms has a more specific meaning.  If a
random variable, measured under some state, can be used to predict the value
of the same variable under another state, then the system is said to be
auto-correlated.

The theory of probability is based on the theory of sets of events.  While a
probability is a numerical value, it is not a variable, but relates to the
chance of certain event being observed.  The classical definition of
probability relates to the expected number of times that any event is
observed, in an infinite number of trials, and thus only relates to past
observations.  The Bayesian definition of probability, which is more useful
in engineering, is defined by the probability that the observer expects to
see an event occur in the next trial.  Given that a trial is performed, the
outcome must be one or more events.  Thus if we are watching for a single
event, and it did not occur, then it's compliment must have occurred.  Thus
the set operators: compliment, union and intersection can be used to define
the probabilities of combinations of events.

Probabilities have some special properties:  they can only take on a value
from 0 to 1, inclusive; the probability of observing an event is always one
minus the probability of not observing the same event and, given a set of
mutually exclusive events covering all outcomes, the sum of the probability
of all of these events is one.

Confidence is a probability value which is chosen by the observer, rather
than being calculated, and is always used in reference to some error or
interval.  Thus the statement `The thickness should be $150\pm10$~mm at
$95\%$ confidence' is correct, while the statement `The thickness should be
$150\pm10$~mm with a $95\%$ probability' is not.  After observing the
thickness we may be able to state that `The thickness is $150\pm10$~mm with
$98.2\%$ probability', which leads to the concept of hypothesis testing. 
The first statement is a hypothesis, while the last is a fact.  If the facts
agree with the hypothesis, then we conclude that the hypothesis is correct
(although there is a chance that we reject a true hypothesis or accept one
which is not true).

Based on these groupings of variables and probabilities, there are three
main groupings of statistical analysis, although they overlap somewhat.  The
first is the analysis of the probabilities of random variables, known as
probability modeling.  The second is the analysis of endogenous random
variables as functions of exogenous random variables, known as statistical
modeling, and the third is the testing of hypotheses, known as statistical
inference, which is determining whether the probability that an event might
be observed compared to the confidence with which we wish it to be observed.

\subsection{Conditional probability}

One of the most powerful ideas in probability theory is the idea of
conditioning.  Once we have observed one random variable it can tell us
something of the probability of observing a particular value for another
variable.  For example, if we observe that it is raining, then we would
except to observe than the temperature is below average.  If we have the
events $A$ and $B$, then we define the conditional probability of observing
$A$ given that we have already observed $B$ (and therefore that the
probability of observing $B$ is greater than zero), as:
\[
P\{A|B\}=\frac{P\{A\cap B\}}{P\{B\}}
\]
or in other words, the probability of observing both $A$ and $B$, weighted
by the probability of observing $B$.  Since probability is independent of
observation, we can assume we measured $A$ first to obtain:
\[
P\{A\cap B\}=P\{A|B\}P\{B\}=P\{B|A\}P\{A\}
\]
If the probability of $A$ is independent of the outcome $B$ then we say that
the two events are independent:
\[
P\{A\cap B\}=P\{A\}P\{B\}=P\{B\}P\{A\}
\]
When we are dealing with conditional probabilities, we refer to the
unconditional probability of observing an event as the marginal probability. 
Given that $B$ has outcomes $1$ to $n$, then we have:
\[
P\{A\}=\sum\limits_{b=1}^n{P\{A|B=b\}P\{B=b\}} 
\]
or that the total probability of observing $A$ is the sum of the
probabilities of observing $A$ conditional on all of the outcomes of $B$.

\subsection{Probability distributions}

The analysis of the probability is complicated by the fact that we cannot
treat discrete and continuous random variables with the same mathematics,
although the concepts are the same.  This is because in a continuous random
variable, the probability of observing any particular value is zero, and the
probabilities have to be handled by integration over some range, while in a
discrete random variable the probability of observing a value between any of
the values is zero, and the probabilities must be handled as summations over
a discrete number of points.  For this reason we talk about probability
density in continuous random variables, and probability mass in discrete
random variables.

Given a random variable $X$, we denote any particular observation as $x$. 
We will begin by considering a discrete random variable, which might take on
any value in the set $A$, with $n$ elements.  No matter what the actual
values in $A$, we can map these values as integers from $1$ to $n$ (for
discrete unbounded sets the same notation applies, except $n=\infty$) in the
same order as in $A$, if $A$ is an ordered set.  The probability of
observing one value $a$, from $1$ to $n$ is a function of the system being
observed and given by:
\[
p_X\left(a\right)=P\{X=a\}
\]
The probability that $x$ is less than or equal to $a$ is given by:
\[
F_X\left(a\right)=P\{X\leqslant a\}=\sum\limits_{x=1}^a{P\{X=x\}}
\]
These two functions are respectively known as the \ac{PMF} and \ac{CDF}. 
Defining probability functions for continuous random variables is a little
more tricky, because the probability of observing a particular value is
zero.  We thus define a \ac{PDF}, which is the limiting probability of
observing a value within a small increment:
\[
f_X\left(x\right)=\mathop{\lim}\limits_{\Delta x\to 0}
	\frac{P\{x\leqslant X < x + \Delta x\}}{\Delta x}
\]
and then define the CDF as:
\[
F_X\left(a\right)=P\{X\leqslant a\}=\int\limits_{-\infty}^a{f_X\left(x\right)dx}
\]

Note that the probability density function is a not constrained to the range
$[0,1]$.  There is a unique relationship between the \ac{PMF} or \ac{PDF} and the CDF,
so that if one is known then there is one and only one possible function
will satisfy the conditions of the other.  For most of the rest of this
discussion we will use the notation for continuous random variables.  In
either case the CDF is a non-decreasing function of $x$, with values from
$0$ to $1$, and is defined as taking on a value of $0$ for undefined values
of $x$.

There are many different functions which satisfy the requirements of either
a \ac{PMF}, \ac{PDF} or CDF, which will not all be described here, since they are well
documented in literature.  Each function takes a number of parameters
(normally one to four parameters), which are known as the distribution
parameters.  For reliability analysis we are mostly concerned with the
normal and lognormal distributions.  However, we will return to them after
discussing some of the general properties of distributions.

Very often we wish to know what value of a random variable corresponds to a
given probability (e.g. for what load is there a 95\% probability that a
beam will fail).  This can be determined using the inverse CDF:
\[
x = F_X^{-1}\left(p\right) \text{ such that } p = F_X\left(x\right)
\]
Also, in many cases we wish to know the probability that a value is greater
than some threshold, which can be calculated using the complementary CDF:
\[
\bar F_X\left(a\right) = P\{X>a\} = \int\limits_a^\infty{f_X\left(x\right)dx} =
	1-F_X\left(a\right)
\]

In cases where we are numerically evaluating functions at probabilities very
close to one it is often better to use a special complementary CDF routine
because of numerical stability.

\subsection{Multivariate probability distributions}

It is also possible to develop probability distributions for more than one
random variable.  These are known as multivariate or joint probability
density functions.  Essentially, these are a probability density function
for a new random variable, which is the set of all possible joint outcomes
of the individual random variables.  We normally use the random variables
$X$ and $Y$ for bivariate distributions and the random vector $\mathbf{X}$
for more than two random variables.  While it is mathematically possible to
develop multivariate distributions for combinations of discrete and
continuous random variables the mathematics is fairly hairy and we will
avoid it.  For discrete random variables the joint \ac{PMF} and CDF can be
defined as:
\begin{align}
  p_{XY}\left(x,y\right)& =P\{X=x\cap Y=y\} \\
  F_{XY}\left(x,y\right)& =P\{X\leqslant x\cap Y\leqslant y\}=
   	\sum\limits_{x_i\leqslant x}{\sum\limits_{y_j\leqslant y}{p_{XY}\left(x_i,y_j\right)}}
\end{align} 
and using the rules for conditional probability outlined above it is
possible to develop a conditional \ac{PMF}:
\[
\begin{split}
p_{X|Y}\left(x|y\right) & = \frac{P\{X=x\cap Y=y\}}{P\{Y=y\}} \\
               & = \frac{p_{XY}\left(x,y\right)}{p_Y\left(y\right)} \\
               & = \frac{p_{XY}\left(x,y\right)}{\sum\limits_x{p_{XY}\left(x,y\right)}}
\end{split} 
\]
A similar scheme applies for continuous random variables:
\begin{align}
F_{XY}\left(x,y\right) & = P\{X\leqslant x\cap Y\leqslant y\}
	= \int\limits_{-\infty}^y{\int\limits_{-\infty}^x{f_{XY}\left(x,y\right)dxdy}} \\
f_{XY}\left(x,y\right) & = \frac{\partial^2F_{XY}(x,y)}{\partial x\partial y} \\
f_{X|Y}\left(x|y\right) & = \frac{f_{XY}\left(x,y\right)}{f_Y\left(y\right)} \\
	& = \frac{f_{XY}\left(x,y\right)}{\int_{-\infty}^\infty{f_{XY}\left(x,y\right)dx}}
\end{align}

The extension of these formulae into $n$ dimensions is fairly obvious.  It
should also be noted that the solution of the CDF requires an $n$-fold
integral over the \ac{PDF}, which, if possible, is tedious mathematically and
costly computationally.

\subsection{Partial descriptors of random variables}

While one or other of the distribution functions completely describe a
random variable, they are normally too cumbersome for general use and
reporting, and so we seek values which summarize the variable.  The most
general summary is the mean, which is the value which we would expect to
take when observed, which is a weighted average of all of the possible
values:
\begin{align}
E\left[X\right]& =\mu_X = \frac{\sum\limits_x{xp_X\left(x\right)}}
	{\sum\limits_x{p_X\left(x\right)}} = \sum\limits_x{x p_X\left(x\right)} \\
E\left[X\right]& =\mu_X = \int\limits_{-\infty}^\infty{x f_X\left(x\right)dx}
\end{align} 
where the $E\left[\bullet\right]$ operator denotes the expected value.  This
is different from the median, which is the value below which half of the
expected outcomes lie, and the mode, which is the point with the highest
probability mass or density.

Now that we know what value to expect, we are also interested in the spread
of the values.  The standard deviation is the expected absolute deviation
from the mean, and is defined as:
\[
\sigma_X = \sqrt{Var\left[X\right]}
\]
where:
\begin{align}
Var\left[X\right] & = E\left[\left(X-\mu_X\right)^2\right]
  = \sum\limits_x{\left(x-\mu_X\right)^2 p_X\left(x\right)} \\
Var\left[X\right] & = E\left[\left(X-\mu_X\right)^2\right]
  = \int\limits_{-\infty }^\infty{\left(x-\mu_X\right)^2 f_X\left(x\right)dx}
\end{align} 
For most distributions in regular use both the mean and standard deviation
are defined by some simple formula involving the distribution parameters. 
It is often convenient to use the coefficient of variation (COV), which is
defined as:
\[
\delta_X = \frac{\sigma_X}{\left|\mu_X\right|}
\]

If we have two random variables $X$ and $Y$ then we often would like to know
if there is any relationship between them.  The simplest measure of a
relationship is determine if, given that we know the deviation of an
observation of one variable from it's mean, we can determine anything about
the deviation of the other variable from it's mean.  This is known as
covariance, and is defined as:
\begin{align}
Cov\left[X,Y\right] & = E\left[\left(X-\mu_X\right)\left(Y-\mu_Y\right)\right]
  = \sum\limits_x{
    \sum\limits_y{\left(x-\mu_X\right)\left(y-\mu_Y\right)p_{XY}\left(x,y\right)}} \\
Cov\left[X,Y\right] & = E\left[\left(X-\mu_X\right)\left(Y-\mu_Y\right)\right]
	 = \int\limits_{-\infty}^\infty{\int\limits_{-\infty}^\infty{
	 	\left(x-\mu_X\right)\left(y-\mu_Y\right)f_{XY}\left(x,y\right)dxdy}}
\end{align} 
The covariance has units which are a product of the units for $X$ and $Y$,
and so it is convenient to define the dimensionless correlation coefficient:
\[
\rho_{XY} = \frac{Cov\left[X,Y\right]}{\sigma_X\sigma_Y}
\]

It is possible to prove that the correlation coefficient has a range
$\left[-1,1\right]$ and takes on a value of $0$ if the variables are
independent, and $1$ or $-1$ if the variables are perfectly correlated.

\subsection{The normal and multinormal distributions}

By far the most widely used distribution is the normal or Gaussian
distribution, the classic bell shaped curve.  The normal distribution is
defined by two parameters, which happen to also be the mean and standard
deviation of any random variable which has a normal distribution.  However,
we will first consider the standard normal distribution which has a mean of
zero and a standard deviation of one.  The \ac{PDF} and CDF are defined (using
traditional notation, including $u$ as the standard normal random variable)
as:
\begin{align}
   U & \sim \operatorname{N}\left(0,1\right) \\
   \varphi\left(u\right) & = \frac{1}{\sqrt{2\pi}}\exp \left(-\frac{u^2}{2}\right) \\
   \Phi\left(u\right) & = \int\limits_{-\infty}^u{\varphi\left(u\right)}du
\end{align} 

There is no closed form solution to the CDF of the normal distribution, and
so it must either be evaluated using a numerical integration technique or
approximated.  In \OP\ there are approximations for the standard normal CDF
and the inverse CDF which are based on a well known approximations using a
technique known as Rational Chebyshev approximations, which are accurate to
at least the computer's accuracy and are evaluated directly, without
iteration.

The generalized form of the normal \ac{PDF} and CDF are:
\begin{align}
	X & \sim \operatorname{N}\left(\mu,\sigma^2\right) \\
	f_X\left(x\right) & = \frac{1}{\sigma\sqrt{2\pi}}
   	\exp\left(-\frac{1}{2}\left(\frac{x-\mu}{\sigma}\right)^2\right) \\
	F_X\left(x\right) & = \Phi\left(\frac{x-\mu}{\sigma}\right)
\end{align} 

If we have more than one random variable then we define the multivariate
normal (or multinormal) distribution as follows:
\begin{align}
\operatorname{E}\left[X_1\right] & = \mu _1 \\
\operatorname{Var}\left[X_1\right] & = \sigma _1^2 \\
\operatorname{Cov}\left[X_2,X_1\right] & = \rho _{2\,1}\sigma _2\sigma _1
\end{align}

We can express this in matrix notation as:
\begin{align}
\mathbf{M} = \begin{bmatrix}
   \mu _1 \\
   \mu _2 \\
   \vdots \\
   \mu _n \\
 \end{bmatrix} &&
\mathbf{D} = \begin{bmatrix}
   \sigma _1 & {} & {} & \text{diag.} \\
   {} & \sigma _2 & {} & {} \\
   {} & {} &  \ddots  & {} \\
   {} & {} & {} & \sigma _n \\
 \end{bmatrix} &&
\mathbf{R} & = \begin{bmatrix}
   1 & {} & {} & \text{sym.} \\
   \rho _{2\,1} & 1 & {} & {} \\
   \vdots  &  \vdots  &  \ddots  & {} \\
   \rho _{n\,1} & \rho _{n\,2} &  \cdots  & 1  \\
 \end{bmatrix} &&
 \mathbf{\Sigma} = \mathbf{DRD}
\end{align}

\begin{align}
  \mathbf{X} & \sim \operatorname{N}\left(\mathbf{M},\mathbf{\Sigma}\right) \\
  f_\mathbf{X}\left(\mathbf{x}\right) & =
		\frac{\left(2\pi\right)^{-n/2}}{\sqrt{\left|\mathbf{\Sigma}\right|}}
		\exp\left(-\frac{1}{2}\left(\mathbf{x}-\mathbf{M}\right)^{\operatorname{T}}
			\mathbf{\Sigma}^{-1}\left(\mathbf{x}-\mathbf{M}\right)\right) \\
  F_\mathbf{X}\left(\mathbf{x}\right) & = 
  	\int\limits_{-\infty}^{x_n}{\cdots\int\limits_{-\infty}^{x_1}
  		{f_\mathbf{X}\left(\mathbf{x}\right) dx_1} \cdots dx_n }
\end{align} 

If we are dealing with correlated standard normal variables (which we will
denote with $\mathbf{z}$ throughout this discussion) then $\mathbf{M}$ is a zero matrix
and $\mathbf{D}$ is the identity matrix, and the distribution reduces to:
\begin{align}
 \mathbf{Z} & \sim \operatorname{N}\left( 0,\mathbf{R}\right) \\
 \varphi _n\left(\mathbf{z},\mathbf{R}\right) & =
 	\frac{\left(2\pi\right)^{-n/2}}{\sqrt{\left|\mathbf{R}\right|}}
 	\exp\left(-\frac{1}{2}\mathbf{z}^{\operatorname{T}}\mathbf{R}^{-1}\mathbf{z}\right)
\end{align}

We can further simplify this equation if we are working in the uncorrelated
standard normal space to:
\[
\begin{array}{*{20}c}
   \mathbf{U}\sim\operatorname{N}(0,1)
 & {\phi_n(\mathbf{u}) = (2\pi)^{{-n\mathord{\left/\vphantom {{ - n} 2} \right.
 \kern-\nulldelimiterspace} 2}} \exp \left( { - \frac{1}
{2}\left\| \mathbf{u} \right\|^2 } \right)}  \\

 \end{array} 
\]

There are no closed form solutions to the $n$-fold integral in the CDF. 
There are however a number of simplifications.  If we are dealing with two
correlated standard normal variables then we can obtain the result:

\[
\begin{gathered}
  \Phi _2 \left( {z_1 ,z_2 ,\rho } \right) = \Phi \left( {z_1 } \right)\Phi \left( {z_2 } \right) + \int\limits_0^\rho  {\text{j}_2 \left( {z_1 ,z_2 ,\rho } \right)d\rho }  \hfill \\
  \text{j}_2 \left( {z_1 ,z_2 ,\rho } \right) = \frac{1}
{{2\pi \sqrt {1 - \rho ^2 } }}\exp \left( { - \frac{{\left( {z_1^2  - 2\rho z_1 z_2  + z_2^2 } \right)}}
{{2\left( {1 - \rho ^2 } \right)}}} \right) \hfill \\ 
\end{gathered} 
\]

The uncorrelated standard normal space is also rotational symmetric, so if
we are interested in the probability contained within a region bounded by a
hyper-plane then it is always possible to rotate that plane so that we can
obtain the marginal CDF in only one dimension.  Since the marginal CDF of
the multinormal distribution is also a normal distribution we can obtain the
following result:

\[
\Phi _n \left( {\mathbf{u}:\beta  - \mathbf{\alpha }^T \mathbf{u} \leqslant 0} \right) = \int\limits_{ - \infty }^\infty  { \cdots \int\limits_\beta ^\infty  {\text{j}_n \left( \mathbf{u} \right)du_1  \cdots } du_n }  = \Phi \left( { - \beta } \right)
\]

\subsection{The lognormal distribution}

Many quantities in engineering, like length and elastic modulus, cannot be
negative, and so require a distribution which does not predict non-zero
probabilities for negative numbers.  For these quantities the lognormal
distribution is used.  It is closely related to the normal distribution
because it is obtained by assuming that the logs of the random variable are
normally distributed.  We will only deal with a single random variable in
this discussion.

The lognormal distribution has two parameters: l and z, and the \ac{PDF} and CDF
are defined as:

\[
\begin{array}{*{20}c}
   {X \sim \operatorname{LN} \left( {\lambda ,\zeta ^2 } \right)} & {f_X \left( x \right) = \frac{1}
{{\sqrt {2\pi } \zeta x}}\exp \left( { - \frac{1}
{2}\left( {\frac{{\ln x - \lambda }}
{\zeta }} \right)^2 } \right)} & {F_X \left( x \right) = \Phi \left( {\frac{{\ln x - \lambda }}
{\zeta }} \right)}  \\

 \end{array} 
\]

l and z are not the mean and standard deviation.  These, and the coefficient
of variation, are given by the following equations:

\[
\begin{array}{*{20}c}
   {\mu  = \exp \left( {\lambda  + {{\zeta ^2 } \mathord{\left/
 {\vphantom {{\zeta ^2 } 2}} \right.
 \kern-\nulldelimiterspace} 2}} \right)} & {\delta  = \sqrt {\exp \left( {\zeta ^2 } \right) - 1} } & {\sigma  = \mu \delta }  \\

 \end{array} 
\]

If we are given the mean and standard deviation (or the coefficient of
variation) it is possible to derive l and z from the following equations:

\[
\begin{array}{*{20}c}
   {\zeta  = \sqrt {\ln \left( {1 + \left( {\frac{\sigma }
{\mu }} \right)^2 } \right)}  = \sqrt {\ln \left( {1 + \delta ^2 } \right)} } & {\lambda  = \ln \mu  - {{\zeta ^2 } \mathord{\left/
 {\vphantom {{\zeta ^2 } 2}} \right.
 \kern-\nulldelimiterspace} 2}}  \\

 \end{array} 
\]


\subsection{Transformation to the standard normal space}

Because we might be working with a number of different probability
distributions, and these might be correlated, the space in which we have to
perform calculations is very complex.  In particular, although we might have
a marginal \ac{PDF} $f(x)$ for each $x$ in our space $\mathbf{x}$, we need the
multivariate \ac{PDF} $f(\mathbf{x})$, which would normally require the
evaluation of an $n$-fold integral over the space of $\mathbf{x}$ (where the
vector $\mathbf{x}$ as $n$ elements).  Mathematically this is a difficult
problem, and it becomes easier to perform calculations when we transform the
variables into the uncorrelated standard normal space.  To achieve this
transformation, we make use of a family of multivariate distributions known
as Nataf distributions.  Nataf distributions take the form:

\[
f_\mathbf{X} \left( \mathbf{x} \right) = \frac{{f_1 \left( {x_1 } \right)f_2 \left( {x_2 } \right) \cdots f_n \left( {x_n } \right)}}
{{\varphi (z_1 )\varphi (z_2 ) \cdots \varphi (z_n )}}\varphi \left( {\mathbf{z},\mathbf{R}_0 } \right) = \varphi \left( \mathbf{u} \right)
\]

Notice that only the marginal \ac{PDF} of each of the random variables appears in
the function, but that the correlation matrix \textbf{R} is replaced by
\textbf{R}0 (the details of this will be discussed momentarily).  Notice
also, that the function assumes that some transformation from $\mathbf{x}$
to $\mathbf{u}$ exists.  Many such transformations might exist, the only
requirement being that the mapping is one-to-one and reversible (i.e. we can
go $\mathbf{x}$ $\rightarrow$ $\mathbf{u}$ $\rightarrow$ $\mathbf{x})$.  The
most useful of these transformations for the Nataf family takes the form:

\[
\begin{gathered}
  \begin{array}{*{20}c}
   {X_i  \sim F_i \left( {x_i } \right)} & {\mu _i  = \operatorname{E} \left[ {X_i } \right]} & {\sigma _i^2  = \operatorname{Var} \left[ {X_i } \right]} & {\rho _{ij}  = \frac{{\operatorname{Cov} \left[ {X_i ,X_j } \right]}}
{{\sigma _i \sigma _j }}}  \\

 \end{array}  \hfill \\
  \mathbf{L}_0 \mathbf{L}_0^{\operatorname{T}}   = \mathbf{R}_0  = \left[ {\rho _{0,ij} } \right] \hfill \\
  \rho _{ij}  = \iint {\frac{{x_i  - \mu _i }}
{{\sigma _i }}\frac{{x_j  - \mu _j }}
{{\sigma _j }}}\varphi _2 \left( {z_i ,z_j ,\rho _{0,ij} } \right)dz_j dz_i  \hfill \\
  \begin{array}{*{20}c}
   {\begin{array}{*{20}c}
   {\mathbf{u} = \mathbf{L}_0^{ - 1} \mathbf{z}}  \\

 \end{array} } & {\mathbf{z} = \left[ {z_i } \right]} & {z_i  = \Phi ^{ - 1} \left( {F_i \left( {x_i } \right)} \right)}  \\

 \end{array}  \hfill \\ 
\end{gathered} 
\]

Where \textbf{L}0 is the lower triangle decomposition of \textbf{R}0,
obtained by Cholesky decomposition.

While the mathematics might appear overwhelming, the mechanics of the
transformation are quite simple.  For each variable $x$ we make use of the
fact that the marginal CDF $F(x)$ provides a one-to-one mapping to
probabilities in a range (0,1).  Using these probabilities we make use of
the inverse standard normal CDF to transform $x$ into a standard normal
variable $z$.  However, these $z$ variables are still correlated, and so we
use a skew transform of the coordinate system, based on the correlation
structure in the standard normal space.  The only mathematically intractable
component is the integral in the equation above (because we have to solve
for $r$0,\textit{ij} and not \textit{ $\rho$ ij}) and Liu and Der Kiureghian
(1989)1 supply a number of closed form approximations for \textit{ $\rho$
}0\textit{,ij} as a function of \textit{ $\rho$ ij} and the distribution
types of \textit{xi} and \textit{xj}.

If \textit{xi} and \textit{xj} are both normal then \textit{ $\rho$
}0\textit{,ij} equals \textit{ $\rho$ ij }by definition.  If \textit{xj} is
lognormal, then \textit{ $\rho$ }0\textit{,ij} is given (exactly) by the
following equations depending on whether \textit{xi} is normal or lognormal:

\[
\begin{gathered}
  \rho _{0,ij}  = \rho _{ij} \frac{{\delta _j }}
{{\zeta _j }} \hfill \\
  \rho _{0,ij}  = \frac{{\ln \left( {1 + \rho _{ij} \delta _i \delta _j } \right)}}
{{\zeta _i \zeta _j }} \hfill \\ 
\end{gathered} 
\]

If we only have normal distributions, then the mathematics is greatly
simplified:

\[
\begin{gathered}
  \begin{array}{*{20}c}
   {X_i  \sim \operatorname{N} \left( {\mu _i ,\sigma _i } \right)} & {\rho _{ij}  = \frac{{\operatorname{Cov} \left[ {X_i ,X_j } \right]}}
{{\sigma _i \sigma _j }}}  \\

 \end{array}  \hfill \\
  \mathbf{L}_0 \mathbf{L}_0^{\operatorname{T}}   = \mathbf{R}_0  = \left[ {\rho _{ij} } \right] \hfill \\
  \begin{array}{*{20}c}
   {\begin{array}{*{20}c}
   {\mathbf{u} = \mathbf{L}_0^{ - 1} \mathbf{z}}  \\

 \end{array} } & {\mathbf{z} = \left[ {\frac{{x_i  - \mu _i }}
{{\sigma _i }}} \right]}  \\

 \end{array}  \hfill \\ 
\end{gathered} 
\]

In addition if we have a lognormal distribution then:

\[
z_i  = \frac{{\ln x_i  - \lambda _i }}
{{\zeta _i }}
\]

This transformation is also reversible, and when going from the uncorrelated
standard normal space to another distribution space we do not need to
calculate the inverse of the lower triangular matrix \textbf{L}0.


\section{Reliability analysis}

Reliability based design has been implemented in various structural design
codes for some time.  In these codes there are two approaches to defining
reliability.  For buildings, bridges and other structures subject to
controlled loading, the members are normally defined to have some `factor of
safety' which can be represented as a probability of failure under the
expected operating conditions.  For dams, towers or other structures subject
to natural loading, the reliability is normally expressed as the worst case
event under which the structure is expected to fail.  Thus a dam might be
capable of routing a 1 in 1000 year flood.  This can also be expressed as a
probability of failure.  While the first type of structure has a `design
life' this is normally not related to the probability of failure.  In the
second kind of structure, the `design life' is the expect life of the
structure -- the structure could fail within a few months of construction
given sufficient loading, or survive to become a tourist attraction some
millennia later.  Most structural systems normally have a reliability higher
than 99.9\%.

Pavements are a different type of structure -- they are designed to
progressively fail under continued loading, and so the `design life' is
normally considered to be the expected time until major rehabilitation will
have to be undertaken.  It is this progressive failure which presents the
biggest problem in terms of characterising reliability within a pavement
structure, because it means that all of the loading needs to be considered
and not just the worst cases.

Traditionally pavement design has focused around the concept of a structural
capacity, defined in terms of the number of standard axles which can be
borne by a pavement until some terminal service condition is reached.  This
structural capacity is compared to the expected number of vehicles which
will traffic the road within it's design life and the equivalent number of
standard axles that this represents.  Reliability has been defined in terms
of the percentage of the length of pavement which will still be serviceable
at the end of the design life: so if a pavement has a reliability of 95\%,
then it is expected that only 5\% of the length will be unserviceable at the
end of the design life.

The main challenge with performing reliability analysis within a pavement
design procedure is not in the actual mathematics, but rather in the
mechanics of the problem.  The first problem is that because pavements fail
progressively, there are interactions between failure modes, which cause
acceleration in the rate of deterioration.  Secondly, because damage is
caused by various random axle loads, all of which, not just the worst case,
must be accounted for, since they all cause some damage.


\subsection{Basic reliability concepts}

There are many different ways of presenting the reliability problem from a
mathematical and statistical perspective -- unfortunately most of the
presentations done in undergraduate design classes are either blatantly
wrong (because of gross over simplification) or subtly misleading.  The
reader is urged to ignore what they have learnt in the past, and focus on
the framework being presented here2.

Despite the name, reliability analysis typically involves determining the
probability of failure or \textit{pf}.  The basis for all calculations in
reliability analysis is a function which defines whether the system has
failed or not.  This is known as the `limit state surface' and denoted by
\textit{g}$(\mathbf{x})$.  By convention, the failure region is the set of
$\mathbf{x}$ such that the limit state function is negative (i.e.  $\{$
$\mathbf{x}$:\textit{g}$(\mathbf{x})$ $\leq$ 0 $\}$ ), where $\mathbf{x}$ is
a vector of random variables, which might include load and resistance
variables.  These random variables have some probability distribution
associated with them, and the joint probability mass function for these
variables is denoted as $f$$\mathbf{X}(\mathbf{x})$, in keeping with
customary statistical notation.  Additionally, we add vectors of parameters
to the two functions, to explicitly show that they depend on some external
information.  The probability of failure can thus be calculated as:

\[
p_f  = P\left\{ {g(\mathbf{x},\theta _g ) \leqslant 0} \right\} = \int_{\left\{ {\mathbf{x}:g(\mathbf{x},\theta _g ) \leqslant 0} \right\}} {f_\mathbf{X} \left( {\mathbf{x},\theta _f } \right)d\mathbf{x}} 
\]

This is graphically represented in Figure 2.1.  In the top half of the
figure we have probability density function, and in the bottom half, the
limit state surface.  Where the limit state surface crosses under the zero
plane, we have a line denoting the intersection.  This line is projected
onto the \ac{PDF}, and the integral of the area outside of this line taken.  In
general it is not possible to provide such figures, because the number of
design variables $x$ in $\mathbf{x}$ is much larger than two.  Also, this
figure is normally plotted as contours in the $\mathbf{x}$ space (as on the
base of the lower plot).



\textbf{Figure 2.1: Graphical representation of failure prediction}

Now that we have some notation, a simple example to illustrate the concept. 
Let us assume we have a bar of material, which is subject to some tensile
force $T$, and which has a tensile load capacity of $C$, which is greater
than zero.  Let us assume the $T$ is normally distributed, with a mean of
10~kN, and a standard deviation of 4~kN.  Let us also assume that $C$ is
lognormally distributed, and independent of $T$, with a mean of 15~kN and a
standard deviation of 2~kN.  We will be using this example for the rest of
the explanations in this section, even though it appears to have little to
do with pavements.

We need to express this information in our notation, but first we need a
limit state surface.  A simple choice would be
\textit{g}$(\mathbf{x})$=\textit{c}-\textit{t}.  In other words, if the
tensile force exceeds the capacity then the bar will fail.  Since the
variables are uncorrelated we have
$f$$\mathbf{X}(\mathbf{x})$=\textit{fC}(\textit{c})\textit{fT}(\textit{t}).

\[
\begin{gathered}
  \begin{array}{*{20}c}
   {\zeta  = \sqrt {\ln \left( {1 + \left( {\frac{2}
{{15}}} \right)^2 } \right)}  = 0.1327} & {\lambda  = \ln 15 - {{\zeta ^2 } \mathord{\left/
 {\vphantom {{\zeta ^2 } 2}} \right.
 \kern-\nulldelimiterspace} 2}}  \\

 \end{array}  = 2.6992 \hfill \\
  \begin{array}{*{20}c}
   {C \sim \operatorname{LN} \left( {\lambda ,\zeta ^2 } \right)} & {f_C \left( c \right) = \frac{1}
{{\sqrt {2\pi } \zeta c}}\exp \left( { - \frac{1}
{2}\left( {\frac{{\ln c - \lambda }}
{\zeta }} \right)^2 } \right)}  \\

 \end{array}  \hfill \\
  \begin{array}{*{20}c}
   {\mu  = 10} & {\sigma  = 4}  \\

 \end{array}  \hfill \\
  \begin{array}{*{20}c}
   {T \sim \operatorname{N} \left( {\mu ,\sigma ^2 } \right)} & {f_T \left( t \right) = \frac{1}
{{\sigma \sqrt {2\pi } }}\exp \left( { - \frac{1}
{2}\left( {\frac{{t - \mu }}
{\sigma }} \right)^2 } \right)}  \\

 \end{array}  \hfill \\
  \begin{array}{*{20}c}
   {f_\mathbf{X} \left( \mathbf{x} \right) = f_C \left( c \right)f_T \left( t \right)} & {g_\mathbf{X} \left( \mathbf{x} \right) = c - t}  \\

 \end{array}  \hfill \\
  p_f  = \int_{\left\{ {\mathbf{x}:g(\mathbf{x}) \leqslant 0} \right\}} {f_\mathbf{X} \left( \mathbf{x} \right)d\mathbf{x}}  = \int\limits_{ - \infty }^\infty  {\int\limits_c^\infty  {f_C \left( c \right)f_T \left( t \right)dtdc} }  \hfill \\ 
\end{gathered} 
\]

Readers with a little knowledge of manipulation of the normal distribution
will know that this integral cannot be solved mathematically.  Thus we have
to seek ways of approximating the solution.  At this point we can either
turn to some method of simplifying the integral though numerical
approximations or solving the integral through simulation.  Let us first
examine an approximate solution, known as the first order reliability method
or FORM.  We will do a very basic analysis because the details of any method
can become time consuming.

We begin by transforming the problem into the uncorrelated standard normal
space, using the method described above.  Since the two variables are
uncorrelated \textbf{R}0 and hence \textbf{L}0 are both the identity matrix,
and $\mathbf{u}$=\textbf{z} so we obtain the following:

\[
\begin{gathered}
  \begin{array}{*{20}c}
   {\mathbf{x} = \left[ {\begin{array}{*{20}c}
   c  \\
   t  \\

 \end{array} } \right]} & {\mathbf{u} = \left[ {\begin{array}{*{20}c}
   {\frac{{\ln c - \lambda }}
{\zeta }}  \\
   {\frac{{t - \mu }}
{\sigma }}  \\

 \end{array} } \right]}  \\

 \end{array}  \hfill \\
  f_\mathbf{X} \left( \mathbf{x} \right) = \text{j}_2 \left( \mathbf{u} \right) \hfill \\
  G\left( \mathbf{u} \right) = g\left( \mathbf{x} \right) \hfill \\
  p_f  = \int_{\left\{ {\mathbf{u}:G(\mathbf{u}) \leqslant 0} \right\}} {\text{j}_2 \left( \mathbf{u} \right)d\mathbf{u}}  \hfill \\ 
\end{gathered} 
\]

This does not appear to have helped, because we still cannot solve the
integral.  However, since we know that we can calculate the integral for a
hyper-plane we take a first order approximation of the new function
G$(\mathbf{u})$ at a point \textbf{u*} on the line G$(\mathbf{u})$=0 :

\[
\begin{gathered}
  G\left( \mathbf{u} \right) \cong G\left( {\mathbf{u}^* } \right) + \nabla G\left( \mathbf{u} \right)^{\operatorname{T}}  \left| {_{\mathbf{u}^\mathbf{*} } \left( {\mathbf{u} - \mathbf{u}^\mathbf{*} } \right)} \right. \propto \beta  - \hat \alpha ^{\operatorname{T}}  \mathbf{u} \hfill \\
  \begin{array}{*{20}c}
   {G\left( {\mathbf{u}^* } \right) = 0} & {\beta  = \hat \alpha ^{\operatorname{T}}  \mathbf{u}^ *  }  \\
   {\hat \alpha  = \left. {\frac{{ - \nabla G\left( \mathbf{u} \right)}}
{{\left\| {\nabla G\left( \mathbf{u} \right)} \right\|}}} \right|_{\mathbf{u}^ *  } } & {\nabla G\left( \mathbf{u} \right)^{\operatorname{T}}   = \nabla g\left( \mathbf{x} \right)^{\operatorname{T}} \mathbf{J}_{\mathbf{U},\mathbf{X}}^{ - 1} }  \\

 \end{array}  \hfill \\
  \mathbf{J}_{\mathbf{U},\mathbf{X}}^{ - 1}  = \mathbf{J}_{\mathbf{X},\mathbf{U}}  = \left[ {\frac{{\partial x_i }}
{{\partial u_j }}} \right] = \operatorname{diag} \left[ {\frac{{\varphi \left( {z_i } \right)}}
{{f_i \left( {x_i } \right)}}} \right]\mathbf{L}_0  \hfill \\
  p_f  \simeq p_1  = \Phi \left( { - \beta } \right) \hfill \\ 
\end{gathered} 
\]

A graphical example is shown in Figure 2.2.  Two problems remain:  to find a
suitable \textbf{u*} and to determine the gradient vector of the limit state
function.  For our example the calculation is simple, but for many
engineering problems it is not possible to differentiate the failure
function with respect all of the random variables involved in the problem. 
It is possible to approximate the gradient vector using a finite difference
method, where we perturb each random variable slightly and calculate the
change in \textit{g}$(\mathbf{x})$, but this requires solving the system for
$n$ random variables each time we need to determine the gradient vector.

Finding \textbf{u*} requires solving an optimisation problem, which is to
find the closest point on $\{$ \textit{G}$(\mathbf{u})$=0 $\}$ to the
origin.  Zhang and Der Kiureghian (1995)3 provide an efficient algorithm,
known as the improved HL-RF algorithm, to solve this problem, which is
expected to converge in around ten iterations.

It is also possible to use a second order approximation to the limit state
function, which is then known as the second order reliability method or
SORM, which will not be described here.



\textbf{Figure 2.2: Example of FORM and SORM approximations}


\subsection{Multiple failure criteria}

If we have a system with more than one failure mode, then we need to
consider if it is a parallel or a series system, or in other words, if it
fails once all of the components have failed, or once one of the components
has failed.  In the these two cases:

\[
\begin{array}{*{20}c}
   {\text{parallel:}} \hfill & {p_f  = \bigcap\limits_{i = 1}^m {P\left\{ {g_i \left( {\mathbf{x},\theta _g } \right)} \right\}} } \hfill  \\
   {\text{series:}} \hfill & {p_f  = \bigcup\limits_{i = 1}^m {P\left\{ {g_i \left( {\mathbf{x},\theta _g } \right)} \right\}} } \hfill  \\

 \end{array} 
\]

Notice that the probability of failure for a series system is not the
minimum of the probability of failure of each component.


\section{Integration using simulation techniques}

The term simulation covers a wide variety of methods in different
mathematical fields, and covers all methodologies where numbers are
generated in the space of possible input variables, and used to calculate
the output of a mathematical representation of a system.  The output can
then be analysed to draw conclusions about the behaviour of the system.  The
theory of simulation is very well understood, and we will use a few minor
shortcuts in the mathematical explanation here for simplicity.  In
statistics, the basic simulation technique is known as Monte Carlo
simulation, and follows a very simple principle.  Random numbers in the
$\mathbf{x}$ space are generated in such a way that the probability density
function of an infinite number of such random numbers matches the function
$f$$\mathbf{X}(\mathbf{x})$.  If these $\mathbf{x}$ values are tested
against some criteria, to indicate the occurrence of the event \textit{E}
(or the corresponding indicator variable \textit{I}), then we obtain an
approximation of the probability of event \textit{E} occurring
(\textit{pE}), by fitting a probability model to all of the outcomes of
\textit{I}.  Monte Carlo simulation is based on the most fundamental
definition of probability -- that the probability of an event is defined by
the proportion of such events occurring in an infinitely large number of
trials.  Because of this, the result obtained is guaranteed to converge to
the correct result, although the there is no guarantee about the rate of
convergence.


\subsection{Generation of random numbers}

To perform sampling, we require the generation of random numbers which
conform to a specified distribution $f$$\mathbf{X}(\mathbf{x})$.  Most
programming languages provide a good algorithm to provide semi-random
numbers $r$ in the range (0,1), and so algorithms for performing this task
will not be discussed here.  We need to perform the transformation into the
desired distribution.  If we are working in the uncorrelated standard normal
space, and we can manipulate two numbers at a time, a simple transformation
exists:

\[
\begin{gathered}
  u_1  = \text{ }\sqrt { - 2\ln \left( {r_1 } \right)} \sin \left( {2\pi r_2 } \right) \hfill \\
  u_2  = \text{ }\sqrt { - 2\ln \left( {r_1 } \right)} \cos \left( {2\pi r_2 } \right) \hfill \\ 
\end{gathered} 
\]

Since we generate the random numbers in the uncorrelated standard normal
space, we need to perform an inverse transformation to our variable space
(i.e. $\mathbf{u}$ $\rightarrow$ \textbf{x)} as described above.


\subsection{Adaptive Importance Sampling}

When we perform simulation, we evaluate the integral in Equation~(1) using
sampling.  Firstly, since we sample over the entire domain of $\mathbf{x}$
we need to replace the integration range by an indicator variable, which is
zero when the condition is false, and one when it is true.  Equation~(1) can
thus be expressed as:

\[
\begin{gathered}
  p_f  = \int {\operatorname{I} (\mathbf{x})f_\mathbf{X} \left( \mathbf{x} \right).d\mathbf{x}}  = \operatorname{E} \left[ {\operatorname{I} (\mathbf{x})} \right] \\ 
  \operatorname{I} (\mathbf{x}) = \left\{ {\begin{array}{*{20}c}
   {1\text{ if }\mathbf{x} \in \left\{ {\mathbf{x}:\bigcup\limits_{j = 1}^m {g(\mathbf{x}) \leqslant 0} } \right\}}  \\
   {\text{0 if }\mathbf{x} \in \left\{ {\mathbf{x}:\bigcup\limits_{j = 1}^m {g(\mathbf{x}) > 0} } \right\}}  \\

 \end{array} } \right. \\ 
  \hat p_f  = \frac{1}
{n}\sum\limits_{i = 1}^n {\operatorname{I} (\mathbf{x}_i )}  \\ 
\end{gathered} 
\]

$\mathbf{x}$\textit{i} is a sample point in $\mathbf{x}$, with $n$ samples.

The problem with simulation is that the accuracy of the estimated failure
probability depends on the square root of $n$, the number of samples, and so
we may need a large number of samples.  One improvement is to increase the
sampling density in areas of interest, known as importance sampling:

\[
\begin{gathered}
  p_f  = \int {\operatorname{I} (\mathbf{x})\frac{{f_\mathbf{X} \left( \mathbf{x} \right)}}
{{h\left( \mathbf{x} \right)}}h\left( \mathbf{x} \right).d\mathbf{x}}  = \operatorname{E} \left[ {\operatorname{I} (\mathbf{x})\frac{{f_\mathbf{X} \left( \mathbf{x} \right)}}
{{h\left( \mathbf{x} \right)}}} \right] \\ 
  \hat p_f  = \frac{1}
{n}\sum\limits_{i = 1}^n {\operatorname{I} (\mathbf{x}_i )\frac{{f_\mathbf{X} \left( {\mathbf{x}_i } \right)}}
{{h\left( {\mathbf{x}_i } \right)}}}  \\ 
\end{gathered} 
\]

\textit{h}$(\mathbf{x})$ is the sampling distribution, which needs to be
non-zero wherever $f$$\mathbf{X}(\mathbf{x})$ and \textit{I}$(\mathbf{x})$
are non-zero.  It now remains to choose a suitable
\textit{h}$(\mathbf{x})$.  In some cases we might have prior knowledge
which leads us to a particular distribution, but in others we need to adapt
to the problem at hand.  Adaptive importance sampling begins by sampling a
small number of points from $f$$\mathbf{X}(\mathbf{x})$ and then uses the
results of this sampling to determine a suitable distribution for
\textit{h}$(\mathbf{x})$, by centering \textit{h}$(\mathbf{x})$ around the
points where \textit{I}$(\mathbf{x})$=1.  This process can be repeated to
continue to improve \textit{h}$(\mathbf{x})$.  If in addition to the
\textit{I}$(\mathbf{x})$ values we also store the \textit{g}$(\mathbf{x})$
values then we can sort these values and use them to obtain a better
estimate of where the failure region lies.

Since the sampling distribution is independent of the probability
distribution of the random variables themselves, we can choose a convenient
distribution (such as a combination of normal and lognormal variables) and
we do not need to match the correlation structure of the random variables
themselves.  This greatly simplifies the transform from $\mathbf{u}$
$\rightarrow$ $\mathbf{x}$.  However, in practice we need to use a Nataf
distribution for $f$$\mathbf{X}(\mathbf{x})$, since we need to combine the
marginal \acp{PDF} and correlation matrix, and so we need to transform our
problem into the standard normal space to compute
$f$$\mathbf{X}(\mathbf{x})$.

If we have multiple failure criteria (which is normally the case in \OP),
then we calculate one sampling distribution for each failure criteria, and
perform an adaptive importance sampling procedure for each failure criteria. 
If we have \textit{m} failure criteria, we combine the sampling
distributions by the simple formula:

\[
\begin{gathered}
  h(\mathbf{x}) = \sum\limits_{j = 1}^m {p_j h_j (\mathbf{x})}  \hfill \\
  p_j  = \frac{{n_j }}
{{\sum\limits_{j = 1}^m {n_j } }} \hfill \\ 
\end{gathered} 
\]

where \textit{nj} is the number of samples from failure criteria \textit{j}.

\chapter{Layered Elastic Calculation Engine}

The layered elastic calculation engine within \OP\ was based on the
\Fortran\ code from ELSYM5M, which was in turn a metric version of the
ELSYM5 program written in the 1960's at the University of California,
Berkeley. The program was written to handle five layers, ten load locations,
with one type of load, and ten evaluation locations. Within \OP\ the
\Fortran\ code was converted directly to \CC, and then run for one load and
one evaluation location, with the \OP\ code performing the load
superposition. This was to enable \OP\ to have more than one type of load.
The mathematics used within the code was, however, that for an $n$-layered
system, and it was only constrained by the size of the statically allocated
data structures.

For reliability analysis (and other tasks, such as back-calculation) the
layered elastic calculation engine must be run a large number of times, and
so it needs to be as efficient as possible. It is also desirable to have an
$n$-layered calculation program, so that we can do more sophisticated
modeling. The converted \Fortran\ code was very difficult to read, and so
also difficult to check and maintain. For these reasons the main calculation
engine of \OP\ was re-written into a \CC\ class, and is capable of
$n$-layer, $n$-load and $n$-location.

The new algorithms used by the code, along with the $n$-layered solution are
described here, although this is not intended to be a comprehensive
discussion on layered elastic theory.  Only the solutions and not the
derivations are given here.  This discussion is based heavily on the summary
of layered elastic theory prepared by Theyse1, with some minor modifications
to better match the actual calculations performed in the code.

\section{Radial coordinate system}

Before we begin any discussion on the evaluation of the layered elastic
system we need to establish some basic definitions.  Within the pavement
structure we report our results in an inverted $(x,y,z)$ coordinate system
(i.e. $z$ is positive down). In this system we have three normal stresses
(\textit{sx},\textit{sy},\textit{sz}), six shear stresses
(\textit{txy}=\textit{tyx},\textit{txz}=\textit{tzx},\textit{tyz}=\textit{tyz}),
three normal strains (\textit{ex},\textit{ey},\textit{ez}), six shear
strains
(\textit{gxy}=\textit{gyx},\textit{gxz}=\textit{gzx},\textit{gyz}=\textit{gyz})and
three components of displacement (\textit{Dx},\textit{Dy},\textit{Dz}).  We
have a \textit{m} number of circular loads, each at some
(\textit{xj},y\textit{j}) position, and with a load \textit{Pj} and load
radius \textit{aj}.  We also have a \textit{o} evaluation locations, each
defined by an (\textit{xk},\textit{yk},\textit{zk}) coordinate.  The layered
system, with $n$ layers, is defined by a number of depths \textit{h}1 though
\textit{hn}+1.  By definition the coordinate system's origin is at the
surface, so \textit{h}1=0.  If we have a semi-infinite lower layer then
\textit{hn}+1= $\infty$ otherwise \textit{hn}+1 is known as the depth to the
rigid base.  Each layer has an elastic modulus \textit{Ei} and Poisson's
ratio \textit{ui}.  By definition the layer \textit{i} consists of all
points where (\textit{hi} $\leq$ $z$ $<$ \textit{hi+1}).

The strains are defined in terms of the stresses and so we will not deal
with them further:

\[
\left[ {\begin{array}{*{20}c}
   {\varepsilon _x }  \\
   {\varepsilon _y }  \\
   {\varepsilon _z }  \\
   {\gamma _{xy} }  \\
   {\gamma _{xz} }  \\
   {\gamma _{yz} }  \\

 \end{array} } \right] = \frac{1}
{{E_i }}\left[ {\begin{array}{*{20}c}
   1 & {} & {} & {} & {} & {\text{sym}\text{.}}  \\
   { - \upsilon _i } & 1 & {} & {} & {} & {}  \\
   { - \upsilon _i } & { - \upsilon _i } & 1 & {} & {} & {}  \\
   0 & 0 & 0 & {2\left( {\upsilon _i  + 1} \right)} & {} & {}  \\
   0 & 0 & 0 & 0 & {2\left( {\upsilon _i  + 1} \right)} & {}  \\
   0 & 0 & 0 & 0 & 0 & {2\left( {\upsilon _i  + 1} \right)}  \\

 \end{array} } \right]\left[ {\begin{array}{*{20}c}
   {\sigma _x }  \\
   {\sigma _y }  \\
   {\sigma _z }  \\
   {\tau _{xy} }  \\
   {\tau _{xz} }  \\
   {\tau _{yz} }  \\

 \end{array} } \right]
\]

We perform the calculations for each load individually, and use the
principle of load superposition (which is valid for any elastic system) to
sum the effects of each load at each evaluation location.  Because the loads
are circular, the problem is therefore axi-symmetric around the centre of
the load, and so we perform the calculations in a radial coordinate system.

In this radial coordinate system we also have three normal stresses
(\textit{sr},\textit{sq},\textit{sz}), six shear stresses
(\textit{trq}=\textit{tqr},\textit{trz}=\textit{tzr},\textit{t
qz}=\textit{tqz}) and three components of displacement
(\textit{Dr},\textit{Dq},\textit{Dz}).  These are shown in Figure 2.1.



\textbf{Figure 2.1: Radial coordinate system}

The following coordinate transforms apply, for the coordinate systems itself
and for stresses within the system:

\[
\begin{gathered}
  \begin{array}{*{20}c}
  {r = \sqrt {x^2  + y^2 } }{z = z} \\
  {\cos \theta  = \frac{x}
{r}}{\sin \theta  = \frac{y}
{r}} \\

 \end{array}  \hfill \\
  \left[ {\begin{array}{*{20}c}
   {\sigma _x }  \\
   {\sigma _y }  \\
   {\sigma _z }  \\
   {\tau _{xy} }  \\
   {\tau _{xz} }  \\
   {\tau _{yz} }  \\

 \end{array} } \right] = \left[ {\begin{array}{*{20}c}
   {\cos ^2 \theta } & {\sin ^2 \theta } & 0 & 0  \\
   {\sin ^2 \theta } & {\cos ^2 \theta } & 0 & 0  \\
   0 & 0 & 1 & 0  \\
   {\cos \theta \sin \theta } & { - \cos \theta \sin \theta } & 0 & 0  \\
   0 & 0 & 0 & {\cos \theta }  \\
   0 & 0 & 0 & {\sin \theta }  \\

 \end{array} } \right]\left[ {\begin{array}{*{20}c}
   {\sigma _r }  \\
   {\sigma _\theta  }  \\
   {\sigma _z }  \\
   {\tau _{rz} }  \\

 \end{array} } \right] \hfill \\ 
\end{gathered} 
\]

When $r$=0 then sin2\textit{q}=�.  If the outward normal to the face of the
element is in the positive coordinate direction, then the positive normal
and shear stresses acting on that face are in the positive coordinate
direction and, if the outward normal to the face is in the negative
coordinate direction, then the positive stresses on that face are in the
negative coordinate direction.


\subsection{Determination of system response}

The theory of evaluating multi-layered elastic systems is relatively well
understood, but will be presented here to help people trying to read and
understand the code.  We need to solve for twelve unknowns: the three normal
stresses, the six shear stresses and the three displacements for the element
depicted in Figure 2.1.  As already noted, for equilibrium in the element
the complementary shear strains must be equal, removing three unknowns.

Because of the radial coordinate system require that there be no
displacement or shear in the direction of rotation, since the radial plane
must remain plane.  This implies \textit{trq}=0,\textit{ tqz}=0 and
\textit{Dq}=0.  We thus have six remaining unknowns.

The four remaining unknown components of stress have to satisfy the
conditions of equilibrium and compatibility.  Timoshenko and Goodier gave
the following equations of equilibrium and compatibility for an axially
symmetrical problem, based on the work of previous researchers such as Lam�,
Clapeyron and Love.

\[
\begin{gathered}
  0 = \frac{{\partial \sigma _r }}
{{\partial r}} + \frac{{\partial \tau _{rz} }}
{{\partial z}} + \frac{{\sigma _r  - \sigma _\theta  }}
{r} \hfill \\
  0 = \frac{{\partial \sigma _z }}
{{\partial z}} + \frac{{\partial \tau _{rz} }}
{{\partial r}} + \frac{{\tau _{rz} }}
{r} \hfill \\
  0 = \frac{{\partial ^2 \sigma _r }}
{{\partial r^2 }} + \frac{{\partial \sigma _r }}
{{r\partial r}} + \frac{{\partial ^2 \sigma _r }}
{{\partial z^2 }} - \frac{2}
{{r^2 }}\left( {\sigma _r  - \sigma _\theta  } \right) + \frac{1}
{{1 + \upsilon }}\left( {\frac{{\partial ^2 \sigma _r }}
{{\partial r^2 }} + \frac{{\partial ^2 \sigma _\theta  }}
{{\partial r^2 }} + \frac{{\partial ^2 \sigma _z }}
{{\partial r^2 }}} \right) \hfill \\
  0 = \frac{{\partial ^2 \sigma _\theta  }}
{{\partial r^2 }} + \frac{{\partial \sigma _\theta  }}
{{r\partial r}} + \frac{{\partial ^2 \sigma _\theta  }}
{{\partial z^2 }} + \frac{2}
{{r^2 }}\left( {\sigma _r  - \sigma _\theta  } \right) + \frac{1}
{{1 + \upsilon }}\left( {\frac{{\partial ^2 \sigma _r }}
{{\partial r^2 }} + \frac{{\partial ^2 \sigma _\theta  }}
{{\partial r^2 }} + \frac{{\partial ^2 \sigma _z }}
{{\partial r^2 }}} \right) \hfill \\
  0 = \frac{{\partial ^2 \sigma _z }}
{{\partial r^2 }} + \frac{{\partial \sigma _z }}
{{r\partial r}} + \frac{{\partial ^2 \sigma _z }}
{{\partial z^2 }} + \frac{1}
{{1 + \upsilon }}\left( {\frac{{\partial ^2 \sigma _r }}
{{\partial r^2 }} + \frac{{\partial ^2 \sigma _\theta  }}
{{\partial r^2 }} + \frac{{\partial ^2 \sigma _z }}
{{\partial r^2 }}} \right) \hfill \\
  0 = \frac{{\partial ^2 \tau _{rz} }}
{{\partial r^2 }} + \frac{{\partial \tau _{rz} }}
{{r\partial r}} + \frac{{\partial ^2 \tau _{rz} }}
{{\partial z^2 }} - \frac{1}
{{r^2 }}\tau _{rz}  + \frac{1}
{{1 + \upsilon }}\left( {\frac{{\partial ^2 \sigma _r }}
{{\partial r^2 }} + \frac{{\partial ^2 \sigma _\theta  }}
{{\partial r^2 }} + \frac{{\partial ^2 \sigma _z }}
{{\partial r^2 }}} \right) \hfill \\ 
\end{gathered} 
\]


They also show that these functions are satisfied by \textit{j}$(r,z)$ ,
which is an Airy function, and known as the stress function, which satisfies
the following equations:

\[
\begin{gathered}
  \sigma _r  = \frac{\partial }
{{\partial z}}\left( {\upsilon \left( {\frac{{\partial ^2 \varphi \left( {r,z} \right)}}
{{\partial r^2 }} + \frac{{\partial \varphi \left( {r,z} \right)}}
{{r\partial r}} + \frac{{\partial ^2 \varphi \left( {r,z} \right)}}
{{\partial z^2 }}} \right) - \frac{{\partial ^2 \varphi \left( {r,z} \right)}}
{{\partial r^2 }}} \right) \\ 
  \sigma _\theta   = \frac{\partial }
{{\partial z}}\left( {\upsilon \left( {\frac{{\partial ^2 \varphi \left( {r,z} \right)}}
{{\partial r^2 }} + \frac{{\partial \varphi \left( {r,z} \right)}}
{{r\partial r}} + \frac{{\partial ^2 \varphi \left( {r,z} \right)}}
{{\partial z^2 }}} \right) - \frac{{\partial \varphi \left( {r,z} \right)}}
{{r\partial r}}} \right) \\ 
  \sigma _z  = \frac{\partial }
{{\partial z}}\left( {\left( {2 - \upsilon } \right)\left( {\frac{{\partial ^2 \varphi \left( {r,z} \right)}}
{{\partial r^2 }} + \frac{{\partial \varphi \left( {r,z} \right)}}
{{r\partial r}} + \frac{{\partial ^2 \varphi \left( {r,z} \right)}}
{{\partial z^2 }}} \right) - \frac{{\partial ^2 \varphi \left( {r,z} \right)}}
{{\partial z^2 }}} \right) \\ 
  \tau _{rz}  = \frac{\partial }
{{\partial r}}\left( {\left( {1 - \upsilon } \right)\left( {\frac{{\partial ^2 \varphi \left( {r,z} \right)}}
{{\partial r^2 }} + \frac{{\partial \varphi \left( {r,z} \right)}}
{{r\partial r}} + \frac{{\partial ^2 \varphi \left( {r,z} \right)}}
{{\partial z^2 }}} \right) - \frac{{\partial ^2 \varphi \left( {r,z} \right)}}
{{\partial z^2 }}} \right) \\ 
  \Delta r =  - \frac{{\left( {1 + \upsilon } \right)}}
{E}\frac{{\partial ^2 \varphi \left( {r,z} \right)}}
{{\partial r\partial z}} \\ 
  \Delta z = \frac{{\left( {1 + \upsilon } \right)}}
{E}\left( {2\left( {1 - \upsilon } \right)\left( {\frac{{\partial ^2 \varphi \left( {r,z} \right)}}
{{\partial r^2 }} + \frac{{\partial \varphi \left( {r,z} \right)}}
{{r\partial r}} + \frac{{\partial ^2 \varphi \left( {r,z} \right)}}
{{\partial z^2 }}} \right) - \frac{{\partial ^2 \varphi \left( {r,z} \right)}}
{{\partial z^2 }}} \right) \\ 
\end{gathered} 
\]

\[
0 = \frac{{\partial ^4 \varphi \left( {r,z} \right)}}
{{\partial r^4 }} + 2\frac{{\partial ^3 \varphi \left( {r,z} \right)}}
{{r\partial r^3 }} + 2\frac{{\partial ^4 \varphi \left( {r,z} \right)}}
{{\partial r^2 \partial z^2 }} + \frac{{\partial ^2 \varphi \left( {r,z} \right)}}
{{r^2 \partial r^2 }} + 2\frac{{\partial ^3 \varphi \left( {r,z} \right)}}
{{r\partial r\partial z^2 }} + \frac{{\partial ^2 \varphi \left( {r,z} \right)}}
{{\partial z^4 }}
\]

Michelow suggested the following solution for the stress function, for layer
\textit{i} in an $n$-layered system, with the addition of an arbitrary
constant parameter \textit{m}:

\[
\varphi _i \left( {r,z,m} \right) = J_0 \left( {mr} \right)\left( {\left( {A_i  + B_i z} \right)e^{mz}  + \left( {C_i  + D_i z} \right)e^{ - mz} } \right)
\]


Where \textit{J}0$(x)$ is a Bessel function of the first kind with order
zero, and $A$, $B$, $C$ and $D$ are constants for each layer which are only
a function of \textit{m}.  This gives the following solution to (1) and
satisfies (2).

\[
\begin{gathered}
  \sigma _z^i  = m^2 J_0 \left( {mr} \right)\left( {\left( {1 - 2\upsilon _i } \right)\left( {B_i e^{mz}  + D_i e^{ - mz} } \right) - m\left( {\left( {A_i  + B_i z} \right)e^{mz}  - \left( {C_i  + D_i z} \right)e^{ - mz} } \right)} \right) \\ 
  \tau _{rz}^i  = m^2 J_1 \left( {mr} \right)\left( {2\upsilon _i \left( {B_i e^{mz}  - D_i e^{ - mz} } \right) + m\left( {\left( {A_i  + B_i z} \right)e^{mz}  + \left( {C_i  + D_i z} \right)e^{ - mz} } \right)} \right) \\ 
  \Delta r^i  = \frac{{1 + \upsilon _i }}
{{E_i }}mJ_1 \left( {mr} \right)\left( {B_i e^{mz}  + D_i e^{ - mz}  + m\left( {\left( {A_i  + B_i z} \right)e^{mz}  - \left( {C_i  + D_i z} \right)e^{ - mz} } \right)} \right) \\ 
  \Delta z^i  = \frac{{1 + \upsilon _i }}
{{E_i }}mJ_0 \left( {mr} \right)\left( {\left( {2 - 4\upsilon _i } \right)\left( {B_i e^{mz}  - D_i e^{ - mz} } \right) - m\left( {\left( {A_i  + B_i z} \right)e^{mz}  + \left( {C_i  + D_i z} \right)e^{ - mz} } \right)} \right) \\ 
  \sigma _r^i  = m^2 J_0 \left( {mr} \right)\left( {\left( {1 + 2\upsilon _i } \right)\left( {B_i e^{mz}  + D_i e^{ - mz} } \right) + m\left( {\left( {A_i  + B_i z} \right)e^{mz}  - \left( {C_i  + D_i z} \right)e^{ - mz} } \right)} \right) \\ 
   - \frac{1}
{r}\frac{{E_i }}
{{1 + \upsilon _i }}\Delta r^i  \\ 
  \sigma _\theta ^i  = 2\upsilon _i m^2 J_0 \left( {mr} \right)\left( {B_i e^{mz}  + D_i e^{ - mz} } \right) + \frac{1}
{r}\frac{{E_i }}
{{1 + \upsilon _i }}\Delta r^i  \\ 
\end{gathered} 
\]

If we consider the a load applied to the surface which is given by:

\[
f\left( r \right) = J_0 \left( {mr} \right)
\]


Then we have the following solutions to (3) at $z$=0:

\[
\begin{gathered}
  \tau _{rz}^1 \left| {_{z = 0}  = 0 = } \right.m^2 J_1 \left( {mr} \right)\left( {2\upsilon _1 \left( {B_1  - D_1 } \right) + m\left( {A_1  + C_1 } \right)} \right) \hfill \\
  \sigma _z^1 \left| {_{z = 0} } \right. = J_0 \left( {mr} \right) = m^2 J_0 \left( {mr} \right)\left( {\left( {1 - 2\upsilon _1 } \right)\left( {B_1  + D_1 } \right) - m\left( {A_1  - C_1 } \right)} \right) \hfill \\ 
\end{gathered} 
\]


and we can formulate the following equations to solve for $A$, $B$, $C$ and
$D$ in the first layer:

\[
\left[ {\begin{array}{*{20}c}
   0  \\
   {\frac{1}
{{m^2 }}}  \\

 \end{array} } \right] = \left[ {\begin{array}{*{20}c}
   m & {2v_1 } & m & { - 2v_1 }  \\
   m & {2v_1  - 1} & { - m} & {2v_1  - 1}  \\

 \end{array} } \right]\left[ {\begin{array}{*{20}c}
   {A_1 }  \\
   {B_1 }  \\
   {C_1 }  \\
   {D_1 }  \\

 \end{array} } \right] = B_1 \left[ {\begin{array}{*{20}c}
   {A_1 }  \\
   {B_1 }  \\
   {C_1 }  \\
   {D_1 }  \\

 \end{array} } \right]
\]

Unfortunately, we only have two equations for the four unknowns, and so we
need to seek further equations in terms of the remaining constants for the
lower layers.  Since we require four constants per layer, but have six
equations for the elastic response, we use only the first four from (3).  We
can develop the following matrix notation for these four equations:

\[
\begin{array}{*{20}c}
   {S_i \left( z \right) = \left[ {\begin{array}{*{20}c}
   {\sigma _z^i }  \\
   {\tau _{rz}^i }  \\
   {u^i }  \\
   {w^i }  \\

 \end{array} } \right] = K_i M_i \left( z \right)D\left( z \right)\left[ {\begin{array}{*{20}c}
   {A_i }  \\
   {B_i }  \\
   {C_i }  \\
   {D_i }  \\

 \end{array} } \right]} & {K_i  = \operatorname{diag} \left[ {\begin{array}{*{20}c}
   { - m^2 J_0 (mr)}  \\
   {m^2 J_1 (mr)}  \\
   {\frac{{1 + \upsilon _i }}
{{E_i }}mJ_1 (mr)}  \\
   { - \frac{{1 + \upsilon _i }}
{{E_i }}mJ_0 (mr)}  \\

 \end{array} } \right]}  \\
   {M_i \left( z \right) = \left[ {\begin{array}{*{20}c}
   \hfill 1 & \hfill {mz + 2v_i  - 1} & \hfill { - 1} & \hfill { - mz + 2v_i  - 1} \\
   \hfill 1 & \hfill {mz + 2v_i } & \hfill 1 & \hfill {mz - 2v_i } \\
   \hfill 1 & \hfill {mz + 1} & \hfill { - 1} & \hfill { - mz + 1} \\
   \hfill 1 & \hfill {mz + 4v_i  - 2} & \hfill 1 & \hfill {mz - 4v_i  + 2} \\

 \end{array} } \right]} & {D\left( z \right) = \operatorname{diag} \left[ {\begin{array}{*{20}c}
   {me^{mz} }  \\
   {e^{mz} }  \\
   {me^{ - mz} }  \\
   {e^{ - mz} }  \\

 \end{array} } \right]}  \\

 \end{array} 
\]

If we consider the layer interface, shown in Figure 2.2, we can match the
stress across a very small layer just above the layer interface to obtain:

\[
\mathop {\lim }\limits_{\varepsilon  \to 0} S_i \left( {h_{i + 1}  - \varepsilon } \right) = S_{i + 1} \left( {h_{i + 1} } \right)
\]



\textbf{Figure 2.2: Layer interface compatibility}

Combining with (5) and rearranging we obtain:

\[
\left[ {\begin{array}{*{20}c}
   {A_i }  \\
   {B_i }  \\
   {C_i }  \\
   {D_i }  \\

 \end{array} } \right] = D^{ - 1} \left( {h_{i + 1} } \right)M_i^{ - 1} \left( {h_{i + 1} } \right)K_i^{ - 1} K_{i + 1} M_{i + 1} \left( {h_{i + 1} } \right)D\left( {h_{i + 1} } \right)\left[ {\begin{array}{*{20}c}
   {A_{i + 1} }  \\
   {B_{i + 1} }  \\
   {C_{i + 1} }  \\
   {D_{i + 1} }  \\

 \end{array} } \right]
\]

\[
X_i \left( z \right) = \frac{1}
{{4\upsilon _i  - 4}}\left[ {\begin{array}{*{20}c}
   \hfill { - mz - 1} & \hfill 1 & \hfill { - mz + 1} & \hfill 1 \\
   \hfill {mz + 4\upsilon _i  - 2} & \hfill { - 1} & \hfill { - mz + 4\upsilon _i  - 2} & \hfill 1 \\
   \hfill {mz + 2\upsilon _i  - 1} & \hfill { - 1} & \hfill {mz - 2\upsilon _i  + 1} & \hfill { - 1} \\
   \hfill {mz - 2\upsilon _i } & \hfill 1 & \hfill {mz - 2\upsilon _i } & \hfill { - 1} \\

 \end{array} } \right]^{\operatorname{T}}  \operatorname{diag} \left[ {\begin{array}{*{20}c}
   1  \\
   1  \\
   {\frac{{1 + \upsilon _{i + 1} }}
{{1 + \upsilon _i }}\frac{{E_i }}
{{E_{i + 1} }}}  \\
   {\frac{{1 + \upsilon _{i + 1} }}
{{1 + \upsilon _i }}\frac{{E_i }}
{{E_{i + 1} }}}  \\

 \end{array} } \right]\left[ {\begin{array}{*{20}c}
   \hfill 1 & \hfill {mz + 2v_{i + 1}  - 1} & \hfill { - 1} & \hfill { - mz + 2v_{i + 1}  - 1} \\
   \hfill 1 & \hfill {mz + 2v_{i + 1} } & \hfill 1 & \hfill {mz - 2v_{i + 1} } \\
   \hfill 1 & \hfill {mz + 1} & \hfill { - 1} & \hfill { - mz + 1} \\
   \hfill 1 & \hfill {mz + 4v_{i + 1}  - 2} & \hfill 1 & \hfill {mz - 4v_{i + 1}  + 2} \\

 \end{array} } \right]
\]

\[
\left[ {\begin{array}{*{20}c}
   {A_i }  \\
   {B_i }  \\
   {C_i }  \\
   {D_i }  \\

 \end{array} } \right] = D^{ - 1} \left( {h_{i + 1} } \right)X_i \left( {h_{i + 1} } \right)D\left( {h_{i + 1} } \right)\left[ {\begin{array}{*{20}c}
   {A_{i + 1} }  \\
   {B_{i + 1} }  \\
   {C_{i + 1} }  \\
   {D_{i + 1} }  \\

 \end{array} } \right]
\]

Which can be used to solve for $A$, $B$, $C$ and $D$ in each layer down to
the $n$th layer.  Unfortunately we still have four unknowns.  If we consider
the case where $z$ $\rightarrow$ $\infty$ then, we know that all of the
system responses must tend to zero, and we can determine from (3) that $A$
$\infty$ , $B$ $\infty$ must tend to zero.  Thus we need to solve for $C$
$\infty$ , $D$ $\infty$ .  The following matrix is obtained for a rigid base
at depth \textit{hn}+1 (this is from the ELSY5M source code):

\[
\left[ {\begin{array}{*{20}c}
   {A_n }  \\
   {B_n }  \\
   {C_n }  \\
   {D_n }  \\

 \end{array} } \right] = \left[ {\begin{array}{*{20}c}
   {\frac{{1 - 4\upsilon _n  - 2mh_{n + 1} }}
{{3 - 4\upsilon _n }}e^{ - 2mh_{n + 1} } } & {\frac{{4{{\left( {2\upsilon _n  - 1} \right)} \mathord{\left/
 {\vphantom {{\left( {2\upsilon _n  - 1} \right)} m}} \right.
 \kern-\nulldelimiterspace} m} - 2mh_{n + 1}^2 }}
{{3 - 4\upsilon _n }}e^{ - 2mh_{n + 1} } }  \\
   {\frac{{2m}}
{{3 - 4\upsilon _n }}e^{ - 2mh_{n + 1} } } & {\frac{{1 - 4\upsilon _n  + 2mh_{n + 1} }}
{{3 - 4\upsilon _n }}e^{ - 2mh_{n + 1} } }  \\
   1 & 0  \\
   0 & 1  \\

 \end{array} } \right]\left[ {\begin{array}{*{20}c}
   {C_\infty  }  \\
   {D_\infty  }  \\

 \end{array} } \right] = R_n \left( {h_{n + 1} } \right)\left[ {\begin{array}{*{20}c}
   {C_\infty  }  \\
   {D_\infty  }  \\

 \end{array} } \right]
\]

In the case where \textit{hn}+1 $\rightarrow$  $\infty$  then this reduces
to:

\[
\mathop {\lim }\limits_{h_{n + 1}  \to \infty } R_n \left( {h_{n + 1} } \right) = \left[ {\begin{array}{*{20}c}
   0 & 0  \\
   0 & 0  \\
   1 & 0  \\
   0 & 1  \\

 \end{array} } \right]
\]

We can combine (4), (6) and (7) to obtain the following system of equations
for $C$ $\infty$ , $D$ $\infty$ which we can solve using a variety of
techniques:

\[
\left[ {\begin{array}{*{20}c}
   0  \\
   {\frac{1}
{{m^2 }}}  \\

 \end{array} } \right] = B_1 \prod\limits_{j = 1}^{n - 1} {\left( {D^{ - 1} \left( {h_{j + 1} } \right)X_j \left( {h_{j + 1} } \right)D\left( {h_{j + 1} } \right)} \right)} R_n \left( {h_{n + 1} } \right)\left[ {\begin{array}{*{20}c}
   {C_\infty  }  \\
   {D_\infty  }  \\

 \end{array} } \right]
\]

In \OP, as with ELSYM5M, we use Cramer's rule, which is a simple solution to
a system of two linear equations:

\[
\begin{array}{*{20}c}
   {\left[ {\begin{array}{*{20}c}
   e  \\
   f  \\

 \end{array} } \right] = \left[ {\begin{array}{*{20}c}
   a & b  \\
   c & d  \\

 \end{array} } \right]\left[ {\begin{array}{*{20}c}
   x  \\
   y  \\

 \end{array} } \right]} & {x = \frac{{\left| {\begin{array}{*{20}c}
   e & b  \\
   f & d  \\

 \end{array} } \right|}}
{{\left| {\begin{array}{*{20}c}
   a & b  \\
   c & d  \\

 \end{array} } \right|}} = \frac{{ed - bf}}
{{ad - bc}}} & {y = \frac{{\left| {\begin{array}{*{20}c}
   a & e  \\
   c & f  \\

 \end{array} } \right|}}
{{\left| {\begin{array}{*{20}c}
   a & b  \\
   c & d  \\

 \end{array} } \right|}} = \frac{{af - ec}}
{{ad - bc}}} & {\left| {\begin{array}{*{20}c}
   a & b  \\
   c & d  \\

 \end{array} } \right| \ne 0}  \\

 \end{array} 
\]

We can make one more simplification to (8) if we notice that:

\[
D\left( {h_j } \right)D^{ - 1} \left( {h_{j + 1} } \right) = \operatorname{diag} \left[ {\begin{array}{*{20}c}
   {me^{mh_j } }  \\
   {e^{mh_j } }  \\
   {me^{ - mh_j } }  \\
   {e^{ - mh_j } }  \\

 \end{array} } \right]\operatorname{diag} \left[ {\begin{array}{*{20}c}
   {\tfrac{1}
{m}e^{ - mh_{j + 1} } }  \\
   {e^{ - mh_{j + 1} } }  \\
   {\tfrac{1}
{m}e^{mh_{j + 1} } }  \\
   {e^{mh_{j + 1} } }  \\

 \end{array} } \right] = \operatorname{diag} \left[ {\begin{array}{*{20}c}
   {e^{ - m\left( {h_{j + 1}  - h_j } \right)} }  \\
   {e^{ - m\left( {h_{j + 1}  - h_j } \right)} }  \\
   {e^{m\left( {h_{j + 1}  - h_j } \right)} }  \\
   {e^{m\left( {h_{j + 1}  - h_j } \right)} }  \\

 \end{array} } \right] = T\left( {h_{j + 1} } \right)
\]

Now that we have solved the system, we turn back to the applied load.  We
have used an applied load which is a Bessel function, which is not very
useful.  We need to solve the system under an applied load of:

\[
f\left( r \right) = \Pi _a \left( r \right) = \left\{ {\begin{array}{*{20}c}
  1{\text{if }r < a} \\
  {\tfrac{1}
{2}}{\text{if }r = a} \\
  \text{0}{\text{if }r > a} \\

 \end{array} } \right.
\]

Which defines a circular load on the surface, centred at the origin, with
radius $a$ and a unit pressure.  Too obtain this load we combine an infinite
number of Bessel function loads, each scaled such that the sum of the loads
matches this applied load.  This is done through a Hankel transform of (9)
(which is also known as a Fourier-Bessel transform).  The result for the
square function $\Pi$ $a(r)$ is well known:

\[
\Pi _a \left( r \right) = \int\limits_0^\infty  {aJ_1 \left( {ma} \right)J_0 \left( {mr} \right)dm} 
\]

Using the principle of load superposition, we can extend (3) for this new
load case:

\[
\begin{gathered}
  \sigma _z^i  = a\int\limits_0^\infty  {m^2 J_1 \left( {ma} \right)J_0 \left( {mr} \right)\left( {\left( {1 - 2\upsilon _i } \right)\left( {B_i e^{mz}  + D_i e^{ - mz} } \right) - m\left( {\left( {A_i  + B_i z} \right)e^{mz}  - \left( {C_i  + D_i z} \right)e^{ - mz} } \right)} \right)dm}  \\ 
  \tau _{rz}^i  = a\int\limits_0^\infty  {m^2 J_1 \left( {ma} \right)J_1 \left( {mr} \right)\left( {2\upsilon _i \left( {B_i e^{mz}  - D_i e^{ - mz} } \right) + m\left( {\left( {A_i  + B_i z} \right)e^{mz}  + \left( {C_i  + D_i z} \right)e^{ - mz} } \right)} \right)dm}  \\ 
  \Delta r^i  = \frac{{1 + \upsilon _i }}
{{E_i }}a\int\limits_0^\infty  {mJ_1 \left( {ma} \right)J_1 \left( {mr} \right)\left( {B_i e^{mz}  + D_i e^{ - mz}  + m\left( {\left( {A_i  + B_i z} \right)e^{mz}  - \left( {C_i  + D_i z} \right)e^{ - mz} } \right)} \right)dm}  \\ 
  \Delta z^i  = \frac{{1 + \upsilon _i }}
{{E_i }}a\int\limits_0^\infty  {mJ_1 \left( {ma} \right)J_0 \left( {mr} \right)\left( {2\left( {1 - 2\upsilon _i } \right)\left( {B_i e^{mz}  - D_i e^{ - mz} } \right) - m\left( {\left( {A_i  + B_i z} \right)e^{mz}  + \left( {C_i  + D_i z} \right)e^{ - mz} } \right)} \right)dm}  \\ 
  \sigma _r^i  = a\int\limits_0^\infty  {m^2 J_1 \left( {ma} \right)J_0 \left( {mr} \right)\left( {\left( {1 + 2\upsilon _i } \right)\left( {B_i e^{mz}  + D_i e^{ - mz} } \right) + m\left( {\left( {A_i  + B_i z} \right)e^{mz}  - \left( {C_i  + D_i z} \right)e^{ - mz} } \right)} \right)dm}  \\ 
   - \frac{1}
{r}\frac{{E_i }}
{{1 + \upsilon _i }}\Delta r^i  \\ 
  \sigma _\theta ^i  = 2\upsilon _i a\int\limits_0^\infty  {m^2 J_1 \left( {ma} \right)J_0 \left( {mr} \right)\left( {B_i e^{mz}  + D_i e^{ - mz} } \right)dm}  + \frac{1}
{r}\frac{{E_i }}
{{1 + \upsilon _i }}\Delta r^i  \\ 
\end{gathered} 
\]

These integrals run from zero, and astute readers will notice that many of
the limit statements in the derivations above do not apply when
\textit{m}=0.  However, the function \textit{J}1(\textit{ma}) is zero when
\textit{m}=0., and so we do not need to solve for the constants in this
case.


\subsection{Gaussian-Legendre Quadrature over products of Bessel functions}

Unfortunately there are no closed form solutions to equations (11), and so
we need to perform numerical integration.  Since the Bessel functions are
smooth and we can compute them at any abscissa we use Gaussian-Legendre
Quadrature to minimise the number of times we need to evaluate the system in
(11), including solving for $C$ $\infty$ , $D$ $\infty$ and \textit{Ai},
\textit{Bi}, \textit{Ci} and \textit{Di} using (8) and (6), since these have
to be performed at each \textit{m} value which we require for the
integration.

All numerical integration techniques make use of the fact that the integral
of a function is the area under the function, and can therefore be
approximated by taking the product of the average of the function over some
range and the width of that range.  At the most basic level we merely
evaluate the function at the centre of the range, but for better
approximations we fit polynomials through a number of points in the range
and establish the average value of the polynomial.

Gaussian-Legendre Quadrature, often called Gaussian integration, makes use
of fact that a polynomial of degree $N$ is defined by $N$+1 coefficients,
and the definite integral of this polynomial is can be determined exactly. 
If we choose both the $x$ positions at which we evaluate the function and
the weights which we assign to each of the $n$ positions, then we have 2$n$
coefficients, which we can fit a polynomial of degree 2$n$-1 exactly.  Some
manipulation shows that it is possible to determine the optimal locations
and the weights for any polynomial independently of the coefficients of the
polynomial.  If we have $n$ Gauss points in the range (-1,1) at points
\textit{xi} and with weights \textit{wi} then the following formula applies:

\[
\int\limits_a^b {f\left( x \right)dx = \frac{{b - a}}
{2}\int\limits_{ - 1}^1 {f\left( {\frac{{a + b}}
{2} + \frac{{b - a}}
{2}\xi } \right)d\xi  \cong \sum\limits_{i = 1}^n {w_i } f\left( {\frac{{a + b}}
{2} + \frac{{b - a}}
{2}\xi _i } \right)} } 
\]

Figure 2.3 shows the two Bessel functions, and it can be seen that the
oscillate about zero, and decay as \textit{m} increases.  However, it is not
these functions which we need to integrate, but rather the product of two
Bessel functions, one scaled by $a$ and the other by $r$.  This results in a
function which oscillates in an even more pronounced fashion around zero
(see Figure 2.4), and therefore becomes difficult to integrate.



\textbf{Figure 2.3: Bessel functions of the first kind}

Since these functions oscillate we need to either approximate them by a high
order polynomial or perform piece wise integration over small ranges.  In
ELSYM5M the later approach is used.  It's integration procedure always uses
four Gauss points, and then breaks the integral at the points the roots of
\textit{J}1$(x)$.  For evaluation radii smaller than $a$ it uses
$x$=\textit{ma} and for those larger than a it uses $x$=\textit{mr}.  As a
result the products always have less than three turning points in the
integration range, and so the 7th order polynomial approximation is fairly
accurate.  However, this means that the integration points, and therefore
\textit{m} values have to be reselected for each evaluation radius outside
of the load, which in turn means that the constants have to be recomputed
for each evaluation location, slowing down the calculations.

In \OP\ the other approach is adopted.  It can be seen from Figure 2.4 that
the number of turning points between any two roots of
\textit{J}1(\textit{ma}) in the products are a function of the ratio $r$/$a$
and are given (approximately) by the formula $r$/$a$+2.  If we wish to fit a
polynomial with at least twice the order of the product of the Bessel
functions (to ensure high accuracy) then we can determine $n$ as follows:

\[
\begin{gathered}
  2\left( {\frac{r}
{a} + 2} \right) = 2n - 1 \hfill \\
  n = \operatorname{int} \left( {\frac{r}
{a} + 3} \right) \hfill \\ 
\end{gathered} 
\]

\textbf{Figure 2.4: Products of Bessel functions}

In the ELSYM5M procedure the number of Gauss points between 0 and the first
root of \textit{J}1(\textit{ma}) for $r$/$a$=10 would 40, while for this
method it is 13.  In addition, if we use max$(r)$ for all of the evaluation
radii then we only need to compute the constants once.  The integral is
calculated over \textit{m} from 0 to $\infty$ and so we need to compute for
successive roots of \textit{J}1(\textit{ma}) until the integral converges or
until it becomes numerically unstable.  The biggest contributor to numerical
instability are the evaluations of exp(\textit{m}max$(z))$ in the evaluation
of the constants at the rigid base.  For a computer using IEEE double
precision floating point numbers, the maximum value of \textit{m}max$(z)$ is
approximately 618.  Thus we can only integrate to \textit{m} $<$
600/max$(z)$ to ensure that we do not have infinite numbers occurring in the
calculations.  In practice we stop the integration once the relative change
from adding a new panel is less than 10-6.


\subsection{Load superposition algorithm}

Because the new calculation engine can be passed any number of loads and
evaluation locations it is important that it handle these intelligently to
avoid doing unnecessary work.  Because the calculations are performed in a
radial coordinate system this means that we need only perform the
calculations once for each radial offset from each load type, and also (by
the principle of superposition) that we can calculate for each load radius
at a unit pressure, then scale the results based on the actual pressure.

To minimise the number of calculations required the new calculation engine
goes through the following procedure: First all of the $z$ positions of the
evaluation locations are collected and a unique sorted list is compiled. 
The layer that each of these $z$ positions falls within is then determined. 
Following this we collect and compile a unique sorted list of all of the
load radii, since the load radius is the most intrusive parameter in the
calculations.  We then loop through the list of load radii performing the
calculations.

To perform the calculations we need to determine the radial coordinates of
each evaluation location from each load (which has the load radius we are
considering).  To do this we calculate the horizontal distance from each
load to each evaluation location and develop another unique sorted list.  We
take the maximum value from this sorted list and use that to calculate the
number of Gauss points, as detailed above.  We then calculate the maximum
\textit{m} value based on the depth to the rigid base, and proceed to
calculate the layer constants for each Gauss point.

Once we have established the integration constants we then loop through the
evaluation radii and depths performing the actual integration.  The result
of this integration is a set of six radial stress, strain or displacement
values for each $(r,z)$ position.  We only perform this calculation for
combinations of $(r,z)$ which are valid (which in practice means that we
build yet another unique sorted list of $z$ positions and use that rather
than the global list.  Once we have establish the radial responses to this
particular load we proceed to add these to Cartesian responses at the
relevant evaluation locations, scaling by the pressure applied.

We then proceed to the through the load radii, repeating the calculations
and accumulating the responses at each location.  Once we are complete we
calculate the principal stresses and also the strains at each evaluation
location.


\section{Calculation of Principal Stresses}

After we have obtained the stresses in a Cartesian coordinate system, we
need to transform these into principal stresses and strains.  This is a well
known problem, of solving the eigenvalues of a matrix.  However, we are
dealing with a 3x3 matrix, and so most numerical techniques are overkill. 
This section describes the procedure used in \OP.  We begin with a general
discussion of solving eigenvalues, based on that from `Numerical Recipes in
C' (XXX: ref), then detailing the optimisations found in the procedure used
in ELSYM5M, and then finally the optimisations used in \OP.


\subsection{Numerical solution of eigenvalues}

An Eigenvalue l, and corresponding (nonzero) eigenvector $\mathbf{x}$, of an
$n$ $\times$ $n$ matrix \textbf{A} are defined as those solving the
following equation:

\[
\begin{gathered}
  \mathbf{Ax} = \lambda \mathbf{x} \\ 
  \left[ {\begin{array}{*{20}c}
   {a_{11}  - \lambda } & {a_{12} } &  \cdots  & {a_{1n} }  \\
   {a_{21} } & {a_{22}  - \lambda } &  \cdots  & {a_{2n} }  \\
    \vdots  &  \vdots  &  \ddots  &  \vdots   \\
   {a_{n1} } & {a_{n2} } &  \cdots  & {a_{nn}  - \lambda }  \\

 \end{array} } \right]\left[ {\begin{array}{*{20}c}
   {x_1 }  \\
   {x_2 }  \\
    \vdots   \\
   {x_n }  \\

 \end{array} } \right] = \left[ {\begin{array}{*{20}c}
   0  \\
   0  \\
    \vdots   \\
   0  \\

 \end{array} } \right] \\ 
\end{gathered} 
\]

Since $\mathbf{x}$ could be scaled by any quantity, we typically require
$\mathbf{x}$ to be a unit normal vector (i.e. it has length one).  The
matrix \textbf{A} can then have at most $n$ distinct eigenvalues.  The
matrix is said to be `degenerate' if two or more of the eigenvalues are
equal.

The eigenvalues can be solved through using the determinant of the equation
above:

\[
\left\| {\mathbf{A} - \lambda \mathbf{I}} \right\| = 0
\]

which is a polynomial of degree $n$.  However, solving using a root finder
is firstly, not optimal, and secondly does not provide the eigenvectors. 
Thus most eigenvalue solutions are by diagonalization of the matrix, because
it is obvious from the equations above that the $n$ eigenvalues of a
diagonal matrix are it's diagonal terms.  These diagonalization procedures
make use of a `similarity transform', which does not alter the eigenvalues
of the matrix:

\[
\begin{gathered}
  \mathbf{D} = \mathbf{X}^{ - 1} \mathbf{AX} \\ 
  \left\| {\mathbf{D} - \lambda \mathbf{I}} \right\| = \left\| {\mathbf{X}^{ - 1} \mathbf{AX} - \lambda \mathbf{I}} \right\| \\ 
   = \left\| {\mathbf{X}^{ - 1} \left( {\mathbf{A} - \lambda \mathbf{I}} \right)\mathbf{X}} \right\| \\ 
   = \left\| {\mathbf{X}^{ - 1} } \right\|\left\| {\left( {\mathbf{A} - \lambda \mathbf{I}} \right)} \right\|\left\| \mathbf{X} \right\| \\ 
   = \left\| {\mathbf{A} - \lambda \mathbf{I}} \right\| \\ 
\end{gathered} 
\]

Another advantage of this scheme is that the eigenvalues are the columns of
the matrix $\mathbf{X}$.  However, finding $\mathbf{X}$ is non-trivial, and
so the procedures make use of an incremental approach, where an approximate
transform is repeatedly applied until the result converges on a diagonal
matrix.

For real symmetric matrices, the oldest and best known transform is the
Jacobi transform, which applies a single rotation in only two of the
dimensions of the matrix, to zero out one of the off diagonal elements.  If
the matrix has only two dimensions, then this rotation only needs to be
applied once (and this is in fact how the principal stresses are solved in a
Mohr Circle formulation).  The matrix $\mathbf{X}$ takes the form:

\[
X_i  = \left[ {\begin{array}{*{20}c}
   1 &  \cdots  & 0 &  \cdots  & 0 &  \cdots  & 0  \\
    \vdots  &  \ddots  &  \vdots  & {} &  \vdots  & {} &  \vdots   \\
   0 &  \cdots  & {\cos \left( \phi  \right)} &  \cdots  & {\sin \left( \phi  \right)} &  \cdots  & 0  \\
    \vdots  & {} &  \vdots  & 1 &  \vdots  & {} &  \vdots   \\
   0 &  \cdots  & { - \sin \left( \phi  \right)} &  \cdots  & {\cos \left( \phi  \right)} &  \cdots  & 0  \\
    \vdots  & {} &  \vdots  & {} &  \vdots  &  \ddots  &  \vdots   \\
   0 &  \cdots  & 0 &  \cdots  & 0 &  \cdots  & 1  \\

 \end{array} } \right]
\]

The largest off diagonal element is normally zeroed first, and the process
repeated, and it can be shown that repeated application is guaranteed to
result in a diagonal matrix.  However, doing matrix multiplication is
expensive, and requires storage space, and so the procedure is applied in
place.  If the rotation applies to the element \textit{apq} then we have the
following equations:

\[
\begin{gathered}
  \begin{array}{*{20}c}
   {a'_{rp}  = a'_{pr}  = \cos \left( \phi  \right)a_{rp}  - \sin \left( \phi  \right)a_{rq} } & {r = 1 \ldots n}  \\
   {a'_{rq}  = a'_{qr}  = \cos \left( \phi  \right)a_{rq}  + \sin \left( \phi  \right)a_{rp} } & {r \ne p,r \ne q}  \\

 \end{array}  \hfill \\
  a'_{pp}  = \cos ^2 \left( \phi  \right)a_{pp}  + \sin ^2 \left( \phi  \right)a_{qq}  - 2\cos \left( \phi  \right)\sin \left( \phi  \right)a_{pq}  \hfill \\
  a'_{qq}  = \cos ^2 \left( \phi  \right)a_{qq}  + \sin ^2 \left( \phi  \right)a_{pp}  + 2\cos \left( \phi  \right)\sin \left( \phi  \right)a_{pq}  \hfill \\
  a'_{pq}  = a'_{qp}  = \left( {\cos ^2 \left( \phi  \right) - \sin ^2 \left( \phi  \right)} \right)a_{pq}  + \cos \left( \phi  \right)\sin \left( \phi  \right)\left( {a_{qq}  - a_{pp} } \right) \hfill \\ 
\end{gathered} 
\]

We solve for $f$ such that \textit{a $\prime$ pq} is zero, giving:

\[
\theta  \equiv \cot \left( {2\phi } \right) \equiv \frac{{\left( {\cos ^2 \left( \phi  \right) - \sin ^2 \left( \phi  \right)} \right)}}
{{2\cos \left( \phi  \right)\sin \left( \phi  \right)}} = \frac{{\left( {a_{qq}  - a_{pp} } \right)}}
{{2a_{pq} }}
\]

There are eight solutions in the first rotation, and so to get the first we
use:

\[
\begin{gathered}
  \theta  = \frac{{\left( {\cos ^2 \left( \phi  \right) - \sin ^2 \left( \phi  \right)} \right)}}
{{2\cos \left( \phi  \right)\sin \left( \phi  \right)}} \\ 
  2\cos \left( \phi  \right)\sin \left( \phi  \right)\theta  = \cos ^2 \left( \phi  \right) - \sin ^2 \left( \phi  \right) \\ 
  2\frac{{\sin \left( \phi  \right)}}
{{\cos \left( \phi  \right)}}\theta  = 1 - \frac{{\sin ^2 \left( \phi  \right)}}
{{\cos ^2 \left( \phi  \right)}} \\ 
  \frac{{\sin ^2 \left( \phi  \right)}}
{{\cos ^2 \left( \phi  \right)}} + 2\frac{{\sin \left( \phi  \right)}}
{{\cos \left( \phi  \right)}}\theta  - 1 = 0 \\ 
  t \equiv \frac{{\sin \left( \phi  \right)}}
{{\cos \left( \phi  \right)}} = \frac{{\operatorname{sgn} \left( \theta  \right)}}
{{\left| \theta  \right| + \sqrt {\theta ^2  + 1} }} = \frac{{\operatorname{sgn} \left( {\frac{{\left( {a_{qq}  - a_{pp} } \right)}}
{{2a_{pq} }}} \right)}}
{{\left| {\frac{{\left( {a_{qq}  - a_{pp} } \right)}}
{{2a_{pq} }}} \right| + \sqrt {\left( {\frac{{\left( {a_{qq}  - a_{pp} } \right)}}
{{2a_{pq} }}} \right)^2  + 1} }} \\ 
   = \frac{{\operatorname{sgn} \left( {a_{qq}  - a_{pp} } \right)2a_{pq} }}
{{\left| {a_{qq}  - a_{pp} } \right| + \sqrt {\left( {a_{qq}  - a_{pp} } \right)^2  + \left( {2a_{pq} } \right)^2 } }} \\ 
\end{gathered} 
\]

where the lesser known closed form solution for a quadratic equation is used
to ensure that we always have the smaller root.  We can thus obtain:

\[
\begin{gathered}
  \cos \left( \phi  \right) = \frac{1}
{{\sqrt {\frac{1}
{{\cos ^2 \left( \phi  \right)}}} }} \\ 
   = \frac{1}
{{\sqrt {\frac{1}
{{\cos ^2 \left( \phi  \right)}}\left( {\sin ^2 \left( \phi  \right) + \cos ^2 \left( \phi  \right)} \right)} }} \\ 
   = \frac{1}
{{\sqrt {t^2  + 1} }} \\ 
  \sin \left( \phi  \right) = t\cos \left( \phi  \right) \\ 
   = \frac{t}
{{\sqrt {t^2  + 1} }} \\ 
\end{gathered} 
\]


We can also simplfy (12), as:

\[
\begin{gathered}
  a'_{pq}  = 0 = \left( {\cos ^2 \left( \phi  \right) - \sin ^2 \left( \phi  \right)} \right)a_{pq}  + \cos \left( \phi  \right)\sin \left( \phi  \right)\left( {a_{pp}  - a_{qq} } \right) \\ 
  a_{qq}  = \frac{{\left( {\cos ^2 \left( \phi  \right) - \sin ^2 \left( \phi  \right)} \right)a_{pq} }}
{{\cos \left( \phi  \right)\sin \left( \phi  \right)}} + a_{pp}  \\ 
  a'_{pp}  = \cos ^2 \left( \phi  \right)a_{pp}  + \sin ^2 \left( \phi  \right)a_{qq}  - 2\cos \left( \phi  \right)\sin \left( \phi  \right)a_{pq}  \\ 
   = \cos ^2 \left( \phi  \right)a_{pp}  + \sin ^2 \left( \phi  \right)\left( {\frac{{\left( {\cos ^2 \left( \phi  \right) - \sin ^2 \left( \phi  \right)} \right)a_{pq} }}
{{\cos \left( \phi  \right)\sin \left( \phi  \right)}} + a_{pp} } \right) - 2\cos \left( \phi  \right)\sin \left( \phi  \right)a_{pq}  \\ 
   = \left( {\sin ^2 \left( \phi  \right) + \cos ^2 \left( \phi  \right)} \right)a_{pp}  + \left( {\frac{{\cos ^2 \left( \phi  \right)\sin \left( \phi  \right) - \sin ^3 \left( \phi  \right) - 2\cos ^2 \left( \phi  \right)\sin \left( \phi  \right)}}
{{\cos \left( \phi  \right)}}} \right)a_{pq}  \\ 
   = a_{pp}  - \frac{{\sin \left( \phi  \right)}}
{{\cos \left( \phi  \right)}}\left( {\cos ^2 \left( \phi  \right) + \sin ^2 \left( \phi  \right)} \right)a_{pq}  \\ 
   = a_{pp}  - ta_{pq}  \\ 
  a'_{qq}  = a_{qq}  + ta_{pq}  \\ 
  a'_{rp}  = \cos \left( \phi  \right)a_{rp}  - \sin \left( \phi  \right)a_{rq}  \\ 
   = \frac{1}
{{\sqrt {t^2  + 1} }}a_{rp}  - \frac{t}
{{\sqrt {t^2  + 1} }}a_{rq}  \\ 
   = {{\left( {a_{rp}  - ta_{rq} } \right)} \mathord{\left/
 {\vphantom {{\left( {a_{rp}  - ta_{rq} } \right)} {\sqrt {t^2  + 1} }}} \right.
 \kern-\nulldelimiterspace} {\sqrt {t^2  + 1} }} \\ 
  a'_{rq}  = {{\left( {a_{rq}  + ta_{rp} } \right)} \mathord{\left/
 {\vphantom {{\left( {a_{rq}  + ta_{rp} } \right)} {\sqrt {t^2  + 1} }}} \right.
 \kern-\nulldelimiterspace} {\sqrt {t^2  + 1} }} \\ 
\end{gathered} 
\]

We also need to establish a stopping criteria for the iterative process. 
The normal procedure is to stop when the sum of the squares of the off
diagonal elements drops below some small threshold.  It is possible to show
that:

\[
\begin{gathered}
  S = \sum\limits_{r \ne s} {a_{rs}^2 }  \hfill \\
  S' = S - 2a_{pq}^2  \hfill \\ 
\end{gathered} 
\]

However, this is not stable under numerical calculation.  If the matrix is
small then \textit{S} can be recalculated after every iteration, otherwise,
this value needs to be completely recalculated if it falls outside the range
0 $<$ \textit{S} $<$ ($n$2-$n$) $\times$ $a$2\textit{pq}.  Otherwise, a
simpler stopping criteria is to use \textit{apq}.


\subsection{Specialized solution for principal stresses}

Using a generalised eigenvalue solution for a three by three stress tensor
is not the best possible solution, since we can make a number of
simplifications by `unrolling' the equations.  The first change is to always
process all three off diagonal elements, rather than just the maximum
element before checking the stopping criteria.  If we process the elements
in a fixed order, there are certain other simplifications which can be made. 
The stress tensor consists of:

\[
\left[ {\begin{array}{*{20}c}
   {\sigma _x } & {\tau _{xy} } & {\tau _{xz} }  \\
   {\tau _{xy} } & {\sigma _y } & {\tau _{yz} }  \\
   {\tau _{xz} } & {\tau _{yz} } & {\sigma _z }  \\

 \end{array} } \right]
\]

although for most of this discussion we will make use of the notation from
the previous section.  We process the elements in the order \textit{txz},
\textit{txy},\textit{tyz} for no particular reason.  Expanding (13) with
\textit{p}=1,\textit{q}=3 we have:

\[
\begin{gathered}
  a'_{11}  = a_{11}  - ta_{13}  \\ 
  a'_{33}  = a_{33}  + ta_{13}  \\ 
  a'_{12}  = {{\left( {a_{12}  - ta_{23} } \right)} \mathord{\left/
 {\vphantom {{\left( {a_{12}  - ta_{23} } \right)} {\sqrt {t^2  + 1} }}} \right.
 \kern-\nulldelimiterspace} {\sqrt {t^2  + 1} }} \\ 
  a'_{23}  = {{\left( {a_{23}  + ta_{12} } \right)} \mathord{\left/
 {\vphantom {{\left( {a_{23}  + ta_{12} } \right)} {\sqrt {t^2  + 1} }}} \right.
 \kern-\nulldelimiterspace} {\sqrt {t^2  + 1} }} \\ 
  a'_{13}  = 0 \\ 
\end{gathered} 
\]


Thus we modify five of the six elements of interest (leaving only one of the
diagonal elements unchanged).  We then move to $a$12 and obtain:

\[
\begin{gathered}
  a'_{11}  = a_{11}  - ta_{12}  \\ 
  a'_{22}  = a_{22}  + ta_{12}  \\ 
  a'_{13}  = {{\left( {a_{13}  - ta_{23} } \right)} \mathord{\left/
 {\vphantom {{\left( {a_{13}  - ta_{23} } \right)} {\sqrt {t^2  + 1} }}} \right.
 \kern-\nulldelimiterspace} {\sqrt {t^2  + 1} }} = {{ - ta_{23} } \mathord{\left/
 {\vphantom {{ - ta_{23} } {\sqrt {t^2  + 1} }}} \right.
 \kern-\nulldelimiterspace} {\sqrt {t^2  + 1} }} \\ 
  a'_{23}  = {{\left( {a_{23}  + ta_{13} } \right)} \mathord{\left/
 {\vphantom {{\left( {a_{23}  + ta_{13} } \right)} {\sqrt {t^2  + 1} }}} \right.
 \kern-\nulldelimiterspace} {\sqrt {t^2  + 1} }} = {{a_{23} } \mathord{\left/
 {\vphantom {{a_{23} } {\sqrt {t^2  + 1} }}} \right.
 \kern-\nulldelimiterspace} {\sqrt {t^2  + 1} }} \\ 
  a'_{12}  = 0 \\ 
\end{gathered} 
\]

Finally we process $a$23 and obtain:

\[
\begin{gathered}
  a'_{22}  = a_{22}  - ta_{23}  \\ 
  a'_{33}  = a_{33}  + ta_{23}  \\ 
  a'_{12}  = {{\left( {a_{12}  - ta_{13} } \right)} \mathord{\left/
 {\vphantom {{\left( {a_{12}  - ta_{13} } \right)} {\sqrt {t^2  + 1} }}} \right.
 \kern-\nulldelimiterspace} {\sqrt {t^2  + 1} }} = {{ - ta_{13} } \mathord{\left/
 {\vphantom {{ - ta_{13} } {\sqrt {t^2  + 1} }}} \right.
 \kern-\nulldelimiterspace} {\sqrt {t^2  + 1} }} \\ 
  a'_{13}  = {{\left( {a_{13}  + ta_{12} } \right)} \mathord{\left/
 {\vphantom {{\left( {a_{13}  + ta_{12} } \right)} {\sqrt {t^2  + 1} }}} \right.
 \kern-\nulldelimiterspace} {\sqrt {t^2  + 1} }} = {{a_{13} } \mathord{\left/
 {\vphantom {{a_{13} } {\sqrt {t^2  + 1} }}} \right.
 \kern-\nulldelimiterspace} {\sqrt {t^2  + 1} }} \\ 
  a'_{23}  = 0 \\ 
\end{gathered} 
\]

We need to recalculate \textit{t} between each stage.

\backmatter
%\SingleSpacing

\nocite{*}
\bibliographystyle{openpave}
\bibliography{openpave}

\begin{acronym}
	\acro{ADDL}{Academic Development and Distribution License}
	\acro{AC}{Asphalt Concrete}
	\acro{PCC}{Portland Cement Concrete}
	\acro{ME}{Mechanistic-Empirical}
	\acro{ESAL}{Equivalent Standard Axle Load}
	\acro{AASHTO}{American Association of State Highway and Transportation Officials}
	\acro{PMS}{Pavement Management System}
	\acro{MEPDG}{Mechanistic-Empirical Pavement Design Guide}
	\acro{LTPP}{Long-Term Pavement Performance}
	\acro{HVS}{Heavy Vehicle Simulator}
	\acro{APT}{Accelerated Pavement Testing}
	\acro{TPMs}{Transition Probability Matrices}
	\acro{FORM}{First-Order Reliability Method}
	\acro{QC/QA}{Quality Control/Quality Assurance}
	\acro{FWD}{Falling Weight Deflectometer}
	\acro{PSI}{Present Serviceability Index}
	\acro{PMF}{Probability Mass Function}
	\acro{PDF}{Probability Density Function}
	\acro{CDF}{Cumulative Distribution Function}
\end{acronym}

%\begin{appendix}
%\appendix
%\include{appendix1}
%\end{appendix}

\end{empfile}
\end{document}
